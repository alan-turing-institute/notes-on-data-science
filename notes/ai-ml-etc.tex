\documentclass[10pt, a4paper, twocolumn]{article}
\usepackage[T1]{fontenc}
\usepackage{lmodern}
\usepackage{amssymb}
\usepackage{amsmath}
\usepackage{microtype}
\usepackage[margin=0.51in]{geometry}
%
\usepackage{hyperref}
\usepackage[citestyle=verbose-ibid]{biblatex}
\addbibresource{references.bib}
%%
%\setlength{\parindent}{1em}
\setlength{\parskip}{\smallskipamount}
%%
\newcommand{\defn}[1]{\textbf{#1}}
\newcommand{\eg}{\emph{Example:}\relax}
%%
%% Artifical Intelligence, Machine Learning, and all that
%% This note was written by James Geddes
%%
\title{Artificial Intelligence, Machine Learning, and All That \\ DRAFT v0.1}
\author{James Geddes}
\begin{document}\maketitle
\begin{quote}
``Intelligence is whatever machines haven't done yet.'' ---Lawrence Tesler
\end{quote}
This note is an opinionated abridgement of what at present constitutes AI,
broadly construed. It is particularly concerned with the characteristics of the
kinds of problems one might think to address with `state-of-the-art' AI\@. It is
not at all concerned with the specifics of particular algorithms, nor with the
niceties that distinguish one field from another. My hope is that the reader
will be better able to answer the question, ``will AI help with my problem?''

We seem to be in the middle of the third golden age of AI: the age of `machine
learning.' The first age (`good old-fashioned AI') lasted from the 1950s until
perhaps the Lighthill report of 1974. The second---continuing the same themes as
the first but having logical reasoning, among other things, as a domain of
application---lasted from about 1980 until the ends of the Alvey Programme, the
Fifth Generation Computing Project, and the Lisp Machine market around the end
of the decade.\footnote{See, for example,
\url{https://en.wikipedia.org/wiki/History_of_artificial_intelligence}. I am
omitting a great deal of history. In particular, the ideas behind `deep
learning' were formed in the early days of AI; and the ideas behind
probabilistic inference arguably date back to the 18th century.} In each of
these ages---and also during the `AI winters' which separated them---there were
genuine advances in AI: in planning and search, automated reasoning and theorem
proving, expert systems, image and speech recognition, and others. Yet there
were also many unrealistic expectations of the breadth of application of these
breakthroughs. I think it is worth trying to understand what it is that made
particular problems tractable and what is still to be done.

The field is extremely broad: it is really many different fields and I am not an
expert in any of them. Comments and corrections on this draft are therefore very
welcome.

\section{Good Old-fashioned AI}

To state the nature of a particular class of problems it is necessary to answer
three questions: \emph{First,} what problem are we trying to solve?
\emph{Second,} what is the universe of possibilities from which a solution might
be drawn? \emph{Third,} what would consititute a solution to that problem?

I think it's fair to say that the problems which have so far been successfully
attacked by AI turn out to have very precise answers to these three
questions. By way of an example, consider the problem of playing a game of
noughts-and-crosses (\emph{a.k.a.}, tic-tac-toe). From any position, we would
like to find a winning move. The universe of possibilities might be described as
`all possible arrangements of noughts and crosses on a $3\times3$ grid and the legal
moves from one to anothger.' A \emph{winning move} is a move such that it is
always possible to reply to any move by one's opponent with a move that either
wins the game immediately, or is another winning move.

In fact, this description is \emph{so} precise, it is immediate how to obtain a
solution: starting with the current position, one writes down all possible
moves, all possible replies to those moves, and so on. Then one labels them as
winning or not, starting at the end of the game and working backwards. Finally,
one choses a winning move. Of course, there is a snag: there are \emph{lots} of
possible games. While it is possible to enumerate all positions in
noughts-and-crosses, there is no hope of doing so for chess or indeed most
problems of interest. That is why the subject is not trivial: clever ways around
this so-called `combinatorial explosion' must be found.

The sort of AI that was developed in the 20th century ``Good old-fashioned AI,'' (or GOFAI) The kinds of problems that were in the
first and second phases of AI in the 20th century were of this ilk: it is
possible to describe, completely and precisely, the set of possibilities from
which one is choosing as well as the the nature of a possibility that would
consitute a `solution.'  Moreover, it is clear how one would solve the problem
if only one had sufficient time and a computer with enough memory.








\section{Automation in general}

\section{The future}

The first thing to say is that there is, at present, no such thing as ``general
artifical intelligence.''

\end{document}
