\documentclass[10pt, a4paper, twocolumn]{article}
\usepackage[T1]{fontenc}
\usepackage{lmodern}
\usepackage{amssymb}
\usepackage{amsmath}
\usepackage{microtype}
\usepackage[margin=0.51in]{geometry}
%
\usepackage{url}
\usepackage[citestyle=verbose-ibid]{biblatex}
\addbibresource{references.bib}
%%
%\setlength{\parindent}{1em}
\setlength{\parskip}{\smallskipamount}
%%
\newcommand{\defn}[1]{\textbf{#1}}
\newcommand{\eg}{\emph{Example:}\relax}
%%
%% Artifical Intelligence
%% This note was written by James Geddes
%%
\title{Artifical Intelligence}
\begin{document}\maketitle

Definitions of ``Artifical Intelligence'' seem to fall into one of two
categories.
\begin{enumerate}
\item\label{item:ai-goals} A very broad \emph{goal}: to make computers do things
  that require ``intelligence'' (whatever that is) or behave in certain
  ways. For example, in their preface, Russell and Norvig~\footcite{ai:rn} say,
  \begin{quote}
    We define AI as the study of agents that receive percepts from the
    environment and perform actions. 
  \end{quote}
  Sometimes a list of the kinds of activies thought to make up intelligence are
  given, usually without any claim of completeness. For example, A recent House
  of Lords report\footcite{ai:lords} quotes the UK Government's Industrial
  Strategy White Paper as explaining AI to be,
  \begin{quote}
    Technologies with the ability to perform tasks that would otherwise require
    human intelligence, such as visual perception, speech recognition, and
    language translation.
  \end{quote}
  To take another example, the EPSRC lists examples of intelligent activies
  as\footnote{\url{https://epsrc.ukri.org/research/ourportfolio/researchareas/ait/}}
  \begin{quote}
  [including] learning and adaptation; sensory understanding and interaction;
  reasoning and planning; search and optimisation; autonomy; and creativity.
  \end{quote}
\item\label{item:ai-methods} An evolving set of \emph{methods:} specific
  computational techniques applied to specific problems. In this category, one
  might hear about optimisation, constraint satisfaction, tree searching,
  ``Bayesian networks,'' ``deep learning,'' and so on. These have the virtue of
  being well-defined techniques. On the other hand, it is sometimes hard to see
  how they get at the problems above, or whether they generalise beyond very
  specific problems.
\end{enumerate}

I think it might be helpful to focus on a third category: an open set of
exemplar, well-defined problems, the solution of which one might interpret as
requiring intelligence.

One very broad class of problems is exemplified by chess (or, if you like,
noughts-and-crosses). It is reasonably straightforward to write a computer
program that could play perfect chess if only it had enough time and memory: the
problem is simply that we it doesn't have enough time or memory.\footnote{Here
  is the program. Given a chessboard position, evaluate the position in the
  following way. Score the position $+1$ if it is a win for you (i.e., you have
  just given mate); $-1$ if it is a win for your opponent; or $0$ if it is a
  stalemate. (This is a mechanical test, simply following the rules.) If none of
  these conditions apply, then do the following: (1) enumerate and score all
  possible moves from this position; (2) if it is your turn from this position
  take the maximum of all these numbers; if it is your opponent's turn, take the
  minimum of all the numbers. (Thus, on your turn, score $+1$ if \emph{any} move
  scores $+1$, and score $-1$ if \emph{all} moves score $-1$.)

  Finally, assuming it is your move, choose any move that leads to a position
  with a score of $+1$.}

and characterised by the property that everything about the
system under study is known. There are only finitely many possible allocations
of a nought, a cross, or a space to nine positions; and in principle it is
straightforward to enumerate all of them. Given such an allocation, it






\section{A brief, partial, recent, history of AI}

\begin{itemize}
\item 1956–1973. Tree searching, games. Attempts at language. Michie,
  McCarthy. Criticism. The Lighthill report. 
\item The 1980s. Logic programming, theorem proving. ``Fifth generation
  computing project.'' 
\item 2000 onwards. ``Machine learning.'' Prediction. ``Deep learning.'' Curve
  fitting. Monte Carlo methods.
\end{itemize}




\printbibliography%
\end{document}
