\documentclass[10pt, a4paper, twocolumn]{article}
\usepackage[T1]{fontenc}
\usepackage{lmodern}
\usepackage{amssymb}
\usepackage{amsmath}
\usepackage{microtype}
\usepackage[margin=0.51in]{geometry}
%%
%\setlength{\parindent}{1em}
\setlength{\parskip}{\smallskipamount}
%%
\newcommand{\defn}[1]{\textbf{#1}}
\newcommand{\eg}{\emph{Example:}\relax}
%%
%% Artifical Intelligence
%% This note was written by James Geddes
%%
\title{Artifical Intelligence}
\begin{document}\maketitle

The term ``Artifical Intelligence'' covers at least three different
\emph{categories} of subject:
\begin{enumerate}
\item\label{item:ai-goals} A general \emph{goal}: to make computers that ``do
  things that require intelligence.''
\item\label{item:ai-problems} An open set of exemplar \emph{problems}, the
  solution of which one might interpret as requiring intelligence.
\item\label{item:ai-methods} An evolving set of \emph{methods}.
\end{enumerate}

Discussions of AI are prone to ambguity about which category is
meant. Furthmore, most discussion start with category~\ref{item:ai-goals}---which
is rather broad and anyway no-one knows what intelligence is---and move on to
category~\ref{item:ai-methods}, which is rather detailed. I think it might be
helpful to focus on the middle category.

\section{A brief, partial, recent, history of AI}

There seem to have been three phases:





\end{document}
