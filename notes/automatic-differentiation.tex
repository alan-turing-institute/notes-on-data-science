\documentclass[11pt, a4paper]{article}
\usepackage[utf8]{inputenc}
\usepackage[T1]{fontenc}
\usepackage{beton}
\usepackage{eulervm}
\usepackage{amsmath}
\usepackage{bm}
\usepackage{microtype}
\usepackage{ellipsis}
\usepackage{booktabs}
\usepackage{graphicx}
\usepackage{color}
%%
\usepackage{minted}
%%
\usepackage[hidelinks]{hyperref}
% \usepackage{siunitx}
%
%\usepackage[medium, compact]{titlesec}
%\usepackage[inline]{asymptote}
%\usepackage{tikz-cd}
\DeclareFontSeriesDefault[rm]{bf}{sbc}
% \usepackage{amssymb}
%% Turing grid is 21 columns (of 1cm if we are using A4)
%% Usually 4 "big columns", each of 4 text cols plus 1 gutter col;
%% plus an additional gutter on the left.
\usepackage[top=2.82cm, bottom=2.82cm, left=1cm, textwidth=11cm, marginparsep=1cm, marginparwidth=7cm]{geometry}
\usepackage[Ragged, size=footnote, shape=up]{sidenotesplus}
%% We used to use a two-column layout
% \setlength{\columnsep}{1cm}
\title{Automatic differentiation}
\author{James Geddes}
\date{\today}
%%
\newcommand{\eg}{\emph{Example:}}
\newcommand{\ie}{\emph{i.e.}}
\newcommand{\isdef}{\mathrel{\stackrel{\text{def}}{=}}}
\newcommand{\set}[1]{\boldmath{#1}}
\newcommand{\setR}{\set{R}}
\hyphenation{anti-sym-met-ric}
%%
\newcommand{\cd}[1]{\mintinline{python}{#1}}
%%
% \usepackage[backend=biber]{biblatex}
% \addbibresource{../cyberdefence.bib}
% \DefineBibliographyStrings{english}{
%   andothers = {\mkbibemph{et\addabbrvspace{al}\adddot}}
% }
%%
\begin{document}
\maketitle

\section{Introduction}


\begin{marginfigure}<0pt>
  \centering
  \includegraphics[width=\marginparwidth]{images/gaussian-plot.pdf}
  \caption{A plot of $\exp(-x^2/2)$.\label{fig:gaussian-plot}}
\end{marginfigure}
Automatic differentiation is the process of programmatically
transforming a program that computes some function into a program that
computes the derivative of that function. Note that the input to
automatic differentiation is a \emph{program}, not a mathematical
expression.

It may be, in a particular application, that only a restricted set of
programming constructs are allowed as inputs to the automatic
differentiation system. For example, one may disallow unbounded loops,
or general recursion, or higher-order functions, and so on. When all
programs are allowed, one sometimes refers to the process as
``differentiable programming.''

\section{Representing programs}

In order to programmatically transform a program, we had better be
able to \emph{represent} that program as data. Consider the function
\begin{equation}
  \label{eq:gaussian-example}
  G(x) = e^{-x^2/2}.
\end{equation}
We might implement this function as a program in Rhombus
as follows:\footnote{Rhombus is a language that is almost, but not quite,
  entirely unlike Python.}
\begin{minted}[xleftmargin=\parindent]{ocaml}
fun gaussian(x):
  math.exp(-x**2/2)
\end{minted}
In order to operate \emph{on} this program we will need to represent
it as a data structure. For now, consider just the expression in the
body of the procedure definition. We \emph{could} choose to represent
this expression just as its surface syntax: that is, as the string
``\cd{math.exp(-x**2/2)}'' (perhaps broken into tokens). However,
there is a great deal of unnecessary complexity in this
representation: As well as a numeric constant and a variable, it
contains a function, an infix operator (with a particular precedence
and associativity), and a prefix operator (also with a particular
precedence). The very first thing an interpreter will do is to convert
expressions like this into a uniform, intermediate representation. It
will be more convenient to work with this intermediate representation
directly.

\begin{marginfigure}<0pt>
  \footnotesize
\begin{minted}{python}
#lang rhombus

class Expr():
  nonfinal

class Lit(val :: Any):
  extends Expr

class Var(name :: String):
  extends Expr

class App(name :: String,
          args :: List.of(Expr)):
  extends Expr
\end{minted}
  \caption{Rhombus class definitions for the ``expression''
    type. These definitions introduce classes for literals, variables,
    and procedure applications, all of which are subtypes of
    expression (indicated by the option ``\cd{extends Expr}.'' The
    operator ``\cd{::}'' introduces a type annotation.\label{fig:class-defs}}
\end{marginfigure}
In general, the meaning of ``expression'' is given by a recursive
definition. An \emph{expression} is either
\begin{enumerate}
\item A \emph{literal} (e.g., the number~2). Given a suitable class
  definition (see figure~\ref{fig:class-defs}) we shall represent
  literal 2 in Rhombus as \cd{Lit(2)}.
\item A \emph{variable} (e.g., $x$). We shall represent the variable
  $x$ in Rhombus as \cd{Var("x")}, where the name of the variable is
  given as a string.
\item A \emph{procedure application} (e.g, $-1$, or $x + 2$, or
  $\exp(3)$). Note that procedure applications cover operators as well
  as what one ordinarily thinks of as functions. We shall represent
  these in Rhombus as, say \cd{App("+", [Lit(1), Lit(2)])}: the name
  of the procedure (as a string) together with a list of
  arguments. The definition is recursive because the arguments must
  themselves be one of the three kinds of expression.
\end{enumerate}

A suitable set of class definitions for these types is shown in
figure~\ref{fig:class-defs}. Written as a data structure, using those
classes, the expression of the example looks like this:
\begin{marginfigure}<0pt>
  \centering
  \includegraphics{images/tree.pdf}
  \caption{A tree, representing the expression denoted by
    ``\cd{gaussian}'' in the main text.\label{fig:expression-tree-gaussian}}
\end{marginfigure}
\begin{minted}{python}
def gaussian:
  App("exp",
      [App("-",
       [App("/",
        [App("**", [Var("x"), Lit(2)]),
         Lit(2)])])])
\end{minted}
Figure~\ref{fig:expression-tree-gaussian} shows the same data
structure as a tree, which is perhaps slightly more
readable as the leaves are simply written as their textual
representations.

This intermediate representation (of a program in the language) is
sometimes known as a ``parse tree,'' or ``syntax tree,'' or
``expression tree.'' It is more convenient than the surface syntax to
when used as input to a program-transforming program but it is not
very convenient to read and write directly. One might hope for a
serialisable format for a parse tree that is both easy for the
computer to parse and easy for people to read and write. We will come
back to this point later.

Notice what is missing from the above. We have developed a data
structure for expressions but not for general programs. We don't have
a way to express bindings, for example (assigment of values to
variables), nor can we express function definitions.





\end{document}

What about the derivative? For example, we may wish to evaluate
\begin{equation}
  \label{eq:partial-example}
  \frac{\partial F(x,y,z)}{\partial x}
\end{equation}
at some particular value of $x$, $y$, and~$z$. The ``chain rule'' is a
general theorem for computing the derivative of the composition of
functions. Suppose that $s$ and $t$ are functions of a single
variable. The chain rule is sometimes written as
\begin{equation*}
  \label{eq:chain-rule}
  \frac{d}{dx} s(t(x)) = \frac{d s}{dt} \frac{dt}{dx}. 
\end{equation*}

However, there is an abuse of notation here. Since $t$ is a function,
it is not clear what is meant by the term $ds/dt$, the derivative of
$s$ with respect to~$t$. What is really meant is ``the derivative of
$s$ with respect to its argument, evaluated at $t(x)$.''  The symbol
$t$ is being used both as a placeholder for the argument and as the
name of a function.

I shall use the following notation. For ``the derivative of $f$ with
respect to its $i$-th argument,'' I shall write $\partial_i f$. With this
notation, the chain rule would be written
\begin{equation}
  \label{eq:chain-rule}
  \frac{\partial F(x)}{\partial x} = (\partial_1s)(t(x)) \, \frac{\partial t}{\partial x}.
\end{equation}

The term $(\partial_1s)(t(x))$ means ``the derivative of the function $s$ with
respect to its first (and, in this example, only) argument, evaluated
at $t(x)$.''

The notation still isn't great. There are two many parentheses for one
thing. And $F(x)$ really ought to mean ``the value of the function $F$
at~$x$'' rather than, as we have used it here, ``the function $F$.''
Part of the problem is that $x$ is doing double duty: as the specific
location at which we are evaluating the derivative and as the variable
with respect to which we are taking the derivative. We shall try to
fix that later.

The chain rule extends to multivariate functions. The derivative of
$s(t(x,y),u(x,t))$ with respect to $x$ is
\begin{equation*}
  \frac{\partial }{\partial x} s\bigl(t(x,y),u(x,t)\bigr) =
  (\partial_1s) \frac{\partial t}{\partial x} 
  + (\partial_2s) \frac{\partial u}{\partial x},
\end{equation*}
and likewise for the derivative with respect to~$y$. In the above, it
is understood that $\partial_1s$, say, is to be thought of as a function of
$u$ and $t$, and evaluated at $u(x)$ and $t(x)$.

If, in the above example, $t$ and $u$ were themselves functions of
functions, then the chain rule applies recursively to
those.

Returning to eq.~\eqref{eq:example}, one obtains
\begin{equation*}
  \begin{aligned}
    \partial F / \partial x &= \partial_1 f, \\
    \partial F / \partial y &= \partial_2 f\,\frac{\partial g}{\partial y}, \\
    \partial F / \partial z &= \partial_2 f\,\frac{\partial g}{\partial z} 
                      + \partial_3f\,\frac{\partial h}{\partial z}. \\
  \end{aligned}
\end{equation*}

Many of the terms in the derivatives of $F$ are zero but the general
pattern is clear. In order to compute the derivative of $F$, one
constructs the same expression tree as in figure~\ref{fig:expression}
and recursively computes the derivatives of each node:
\begin{enumerate}
\item If the node is a variable, then its value is the value of that
  variable (as before) and its derivative with respect to some
  variable is either 0 (if the variables are different) or 1 (if they
  are the same). 
\item Otherwise, the node is a function application. Compute the values
  of the arguments, and the values of their derivatives. Now multiply
  the derivative of this function with respect to each of \emph{its}
  arguments by the corresponding derivative of the arguments, and sum. 
\end{enumerate}

The complete set of derivatives of $F$ can thus be computed in a
single, depth-first traversal of the tree. Say the expression graph
has $V$ nodes and $E$ edges; and there are $N$ variables. The cost of
the original expression is $O(V)$, since one performs one function
evaluation for each node. What is the computational cost of the
calculation of the derivatives?

Suppose we are computing the derivative with respect to~$x$. Consider
some node, $f$, encountered during the calculation, having $n$
arguments. We must evaluate $f$, as before, and, in addition, each of
its derivatives,~$\partial_i f$ (for $i=1,\dotsc,n$). Then there are
$n$ multiplications of the $\partial_i f$ with the $x$-derivatives of the
arguments (as well as the sum).

Overall, the total number of arguments over all nodes is~$E$. So we
have a cost of $O(V+E)$ for the evaluation of the function and its
derivatives, along with $O(V+E)$ multiplications. That was to
calculate the derivative with respect to a single variable. The
calculation of the $f$s and the $\partial_i f$s can be done once, but the
multiplications must be repeated for each variable and thus are
$O(N(V+E))$.

The above approach is called ``forward mode automatic
differentiation.'' Roughly, the story of forward mode automatic
differentiation is as follows: ``for each node, compute the derivative
with respect to $x$, and propagate this up towards the root of the
expression.''

However, it turns out that there is an alternative approach. The
alternative approach, known as ``reverse mode,'' is ``starting at the
root of the expression tree, for each node, compute the $\partial_i$, the
derivative with respect to that node's arguments, and propagate these
downwards towards the leaves.'' It is necessary to precompute the
values of the nodes beforehand (in order to be able to evaluate each
of the $\partial_i f$). The propagation step is then $O(E)$ (in
multiplications). 










\section{Outline of talk}

\begin{enumerate}
\item Composition of functions of a single variable.
\item 
\end{enumerate}

\end{document}

% Local Variables:
% TeX-command-extra-options: "-shell-escape"
% End:
