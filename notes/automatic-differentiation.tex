\documentclass[11pt, a4paper]{article}
\usepackage[utf8]{inputenc}
\usepackage[T1]{fontenc}
\usepackage{beton}
\usepackage{eulervm}
\usepackage{amsmath}
\usepackage{bm}
\usepackage{microtype}
\usepackage{ellipsis}
\usepackage{booktabs}
\usepackage{graphicx}
\usepackage{color}
\usepackage{hyperref}
% \usepackage{siunitx}
%
%\usepackage[medium, compact]{titlesec}
%\usepackage[inline]{asymptote}
%\usepackage{tikz-cd}
\DeclareFontSeriesDefault[rm]{bf}{sbc}
% \usepackage{amssymb}
%% Turing grid is 21 columns (of 1cm if we are using A4)
%% Usually 4 "big columns", each of 4 text cols plus 1 gutter col;
%% plus an additional gutter on the left.
\usepackage[top=2.82cm, bottom=2.82cm, left=1cm, textwidth=11cm, marginparsep=1cm, marginparwidth=7cm]{geometry}
\usepackage[Ragged, size=footnote, shape=up]{sidenotesplus}
%% We used to use a two-column layout
% \setlength{\columnsep}{1cm}
\title{Automatic differentiation}
\author{James Geddes}
\date{\today}
%%
\newcommand{\eg}{\emph{Example:}}
\newcommand{\ie}{\emph{i.e.}}
\newcommand{\isdef}{\mathrel{\stackrel{\text{def}}{=}}}
\newcommand{\set}[1]{\boldmath{#1}}
\newcommand{\setR}{\set{R}}
\hyphenation{anti-sym-met-ric}
%%
% \usepackage[backend=biber]{biblatex}
% \addbibresource{../cyberdefence.bib}
% \DefineBibliographyStrings{english}{
%   andothers = {\mkbibemph{et\addabbrvspace{al}\adddot}}
% }
%%
\begin{document}
\maketitle

Here is an example of a certain class of problems. Suppose we are
given functions $f\colon \setR^3 \to \setR$,
$g\colon \setR^2\to\setR$, and $h\colon \setR\to\setR$, in such a way that
we are able to evaluate these functions and their derivatives at any
point. In addition, suppose we are given some expression involving
composititions of these functions; for example,
\begin{equation}
  F(x, y, z) = f(x, g(y, z), h(z)).
\label{eq:example}\end{equation}
The problem is to evaluate $F$ and \emph{its} derivatives at some
point.

What does it mean to say that we are ``given'' some expression? That
is, what is the meaning of eq.~\eqref{eq:example}? It is a tree, shown
on the right in figure~\ref{fig:example-as-tree}.
\begin{marginfigure}

\end{marginfigure}



\end{document}
