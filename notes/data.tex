\documentclass[10pt, a4paper, twocolumn]{article}
%\usepackage{graphicx}
%\usepackage{textcomp}
%\usepackage{amssymb}
%\usepackage{minted}
\usepackage[T1]{fontenc}
\usepackage{concrete}
\usepackage{euler}
\usepackage{amsmath}
\usepackage{amssymb}
\usepackage{mathrsfs}
%\usepackage{fontspec}
\usepackage[margin=0.51in]{geometry}
%\setlength{\parskip}{\smallskipamount}
\usepackage{parskip}
\usepackage{booktabs}
\usepackage[sf, bf, medium, compact]{titlesec}
%%
\usepackage[style=authoryear]{biblatex}
\addbibresource{notes.bib}
%%
%% Table dum and Table dee (false and true)
\newcommand{\dum}{\textbf{\textsf{f}}}
\newcommand{\dee}{\textbf{\textsf{t}}}
\DeclareMathOperator{\schema}{sch}
\DeclareMathOperator{\extension}{ext}
%%
\author{James Geddes}
\date{March 2022}
\title{Data (INCOMPLETE)}
\setcounter{tocdepth}{2}
\begin{document}\maketitle
\tableofcontents
\section{Introduction}
\begin{table*}[!ht]
\centering\small
\begin{tabular}{@{}rllrrrrlr@{}}
  \toprule
 & species & island & bill length & bill depth & flipper length & body mass & sex & year \\ 
 & & & (mm) & (mm) & (mm) & (g) & &  \\ 
  \midrule
  1 & Gentoo & Biscoe & 46.2 & 14.5 & 209 & 4800 & female & 2007 \\ 
  2 & Chinstrap & Dream & 50.9 & 19.1 & 196 & 3550 & male & 2008 \\ 
  3 & Adelie & Dream & 44.1 & 19.7 & 196 & 4400 & male & 2007 \\ 
  4 & Gentoo & Biscoe & 42.6 & 13.7 & 213 & 4950 & female & 2008 \\ 
  5 & Adelie & Biscoe & 40.6 & 18.8 & 193 & 3800 & male & 2008 \\ 
  \bottomrule
\end{tabular}
\caption{Five rows sampled at random from the 344 rows of the ``Palmer Penguins''
  dataset (\cite{palmerpenguins}).\label{tbl:penguins}}
\end{table*}

Table~\ref{tbl:penguins} shows a few rows from a publicly-available dataset. It
is presented as a table, perhaps the most common form of ``structured'' data. But
there is a great deal of information \emph{not} explicitly included in this
table. For example:
\begin{enumerate}
\item{There are, in the world, things called \emph{penguins}.}
\item{The rows of this table describe the properties of an observational unit;
  and that unit is a penguin.}
\item{No individual penguin could occur more than once in any possible version
  of this table.}
\item{Penguins come in distinct \emph{species} and \emph{sexes}.}
\item{\emph{Islands} are also a thing in the world.}
\item{Each penguin stands in a certain spatial relation to one and only one
  island.}
\item{A \emph{flipper length} is a kind of \emph{physical quantity;} specifically, one
  with the dimensions of \emph{length} and the units of \emph{millimetres}.}
 \end{enumerate}
These facts could not be deduced by mere inspection of the data in the
table. The word ``penguin'' doesn't even appear in the table! The concepts of
``species'' and ``sex'' (fundamental characteristics of penguins) are different
from the concept ``island'' (which would seem to be contingent) but these two
things enter the table in the same way. Neither length nor mass is an integer,
even though the values in the table are all integral.Some values in the full
table are missing. Presumably a missing value indicates that the corresponding
property was not measured or is otherwise unknown, not that the property has no
value.

All of the above is information beyond that which is available from the table
and yet it is required to actually make use of this table. There's a
tongue-in-cheek observation in the data science community that ``80\,\% of one's
time is spent cleaning the data.'' Given that most of the information
\emph{about} the data is not \emph{in} the data perhaps this observation is
understandable. The point of this note is to try to understand what we can say
about this ``missing'' information: the data that you need to be able to
understand the data. In other words, what \emph{is} data? In still other words,
what is our \emph{theory} of data?

Perhaps data are \emph{facts}. Then a theory of data would be available if we
had a theory of facts. Well, I'm not sure what facts are. But one thing they
might be is \emph{true propositions}. And we do have a theory of propositions:
namely, logic.\footnote{Actually---there are many logics, so perhaps there are
multiple theories of data.}

Two theories of data based on logic are widely used (though not, in my
experience, well-understood by data scientists): the \emph{relational model} and
\emph{ontologies} (specifically, those ontologies described by a ``description
logic''). The relational model provides the underpinnings of relational
databases \parencite{codd1970relational} whereas ontologies provide the basis
for the `semantic web' \parencite{DBLP:journals/corr/abs-1201-4089}. But there
is clearly something unsatisfactory about having more than one theory of data.
Presumably neither is sufficient because each is missing something.

The literature on these two theories tends to focus on the computational
complexity of reasoning about the data and not so much on what the data
``mean.''  Indeed, both theories are intentionally impoverished compared to (at
least some varieties of) logic in order that they be \emph{decidable}; and
ideally, decidable in something faster than exponential time. In this note, I
propose to completely ignore questions of time and space complexity and mostly
ignore questions of decidability.\footnote{Frankly, you'll be lucky if I worry
about completeness.}

\section{What ought a theory of data cover?}

This section is currently a collection of random thoughts. 

\begin{enumerate}
  \item There seems to be a distinction between ``brute'' facts---that is, rows in the
table---and facts about which brute facts are possible. An example of the former
is ``there existed, in 2007, a female Gentoo penguin on the island of Briscoe.''
An example of the latter is, ``every penguin is either male or female.''

In logic, brute facts are called ``ground.'' Is that the same distinction?

In description logic, a distinction, possibly informal, is made between
assertions in the ``A-box'' and terminological facts in the ``T-box''. Is that
the same distinction?

\item Given a table, one might ask about the status of possible rows that are,
  however, not in the table. On one view, the ``closed-world'' assumption, any
  row that is not in the table corresponds to a proposition that is false. On
  another view, the ``open-world'' assumption, a row not in the table
  corresponds to a proposition that is not known to be true or false.

  Typically, the relational model takes the closed-world view (although
  real-world databases are a lot less scrupulous about this) whereas ontological
  models take the open-world view.

\item There appear to be \emph{individuals}: for example, individual
  penguins. And these individuals instantiate certain \emph{kinds}: for example,
  the kind \emph{penguin}. Then again, penguins have certain properties: for
  example, being of a particular species, such as Gentoo. Do properties
  instantiate kinds? Is ``Gentoo'' an individual? What are our ontological
  commitments here? On this topic, see, eg, \textcite{lowe2005four}.

\end{enumerate}


\section{Preliminaries}

Let $X, Y, \dots, Z$ be sets. Then a \emph{relation} over $X, Y, \dots, Z$ is a
subset of $X\times Y\times\dotsm\times Z$.

\emph{Example:} The set $R_\leq \equiv \{(x, y)\in\mathbb{R}\times\mathbb{R} \mid x\leq y\}$ is a
relation. (We could have called the set $\leq$ but writing $(1, 2)\in\leq$ is a bit odd.)

A relation over $X\times X$ is called a \emph{binary relation}. Certain
kinds of binary relation have special names. A binary relation $R$ is said to
be:
\begin{description}
\item[Reflexive] if $(x, x)\in R$ for all $x\in X$;
\item[Symmetric] if $(x, y)\in R$ implies $(y,x)\in R$;
\item[Antisymmetric] if $(x, y)\in R$ and $(y, x)\in R$ imply $x = y$;
\item[Transitive] if $(x, y) \in R$ and $(y, z) \in R$ imply $(x, z) \in R$.
\end{description}

\emph{Example:} The relation $\leq$ is reflexive, antisymmetric, and transitive.

\emph{Example:} A relation over the ``unary product'' $X$ is just a subset of
$X$.

\emph{Example:} A map, $f:X\to Y$ is a relation over $X\times Y$. Specifically, it is
one such that for all $x\in X$ there is some $y\in Y$ for which $(x, y)\in f$.

\emph{Example:} We don't normally talk about the Cartesian product of \emph{no}
sets. However, it is natural\footnote{At least, natural in the category
\textbf{\textsf{Set}}.} to define it to be $\{\emptyset\}$, the set containing a single
element (which we might as well take to be the empty set). Thus there are two
``nullary'' relations: one containing no elements; and one containing a single
(empty) element.\footnote{\textcite{darwendate2006ttm} call these ``table dum''
and ``table dee.'' I forget which is which.}

Why are we interested in relations? One good reason is that they can be
interpreted as the extensions of \emph{propositions}: that is, of claims that
could be true or false. For example, we might interpret $(x, y)\in R_\leq$ as the
proposition ``$x$ is less than or equal to $y$.'' Thus, $(1, 2)\in R_\leq$ because
the claim that $1\leq 2$ is true; contrariwise, $(3, 2)\notin R_\leq$. When we interpret
things this way, we often write the proposition as, for example, $R_\leq(1, 2)$
(noting that this may be true or false).

See \textcite{cameron1999sets} for more on set theory (from which much of the
above is taken).


\section{The relational model}

Table~\ref{tbl:penguins} looks rather like the sort of thing one finds in a
relational database so perhaps the relational model is the place to start.

I find most treatements of the relational model to be \emph{extremely}
confusing. They move from implementation-level details to very formal
mathematics without clarifying why the one and not the other is necessary in the
moment. They switch from using names for attributes to using position; and move
between algebra and logic. Perhaps the issue is there is not a single
model. This section is my attempt to describe some particular version of the
theory.

The main feature of a table such as table~\ref{tbl:penguins} is that it consists
of a set of \emph{tuples} (or rows); and the values in corresponding positions
in each tuple (that is, in the same column) have the same ``type.''  The set of
tuples really is supposed to be a \emph{set}---no two rows can be identical. (In
production systems this rule is often violated for efficiency reasons.)

To try to capture the idea that ``values in the same column have the same
type,'' let $\mathscr{D}_1, \mathscr{D}_2, \dots, \mathscr{D}_n$ be a finite
collection of sets called \emph{domains}. The point of these domains is to be
the ``carriers'' of the types of data; that is, each value in a table will be an
element of one of the domains and all values in the same column will be elements
of the same domain.\footnote{In Codd's version, and in SQL, a value may also be
  the special value \texttt{NULL}. The addition of \texttt{NULL} significantly
  complicates the relational theory.}

Already there is trouble: The theory offers very little advice on how these
domains are supposed to be chosen. For example,
\textcite{abiteboul1995foundations} say (in section 3.1):
\begin{quote}
  For most of the theoretical development, it suffices to use the same domain of
  values for all of the attributes. Thus we now fix a countably infinite set
  \textbf{dom} [...].
\end{quote}
These authors are happy to assume a single universe holding all possible values
(so long as there are not more than countably many of them\footnote{Actually,
Date believes that they should be finitely many.}).\footnote{Which at
least seems consistent with first-order logic. There, if one wants to distinguish
different kinds of individual, one does so with suitable predicates.} On the
other hand, they also say, a few paragraphs earlier,
\begin{quote}
  The entries of tuples are taken from sets of constants, called domains, that
  include, for example, the sets of integers, strings, and Boolean values.
\end{quote}
That sounds like there ought to be a more formal collection of domains, possibly
various mathematical objects. In practice, in most production systems, the
domains are usually taken to be the ``built-in types'' common to progamming
languages: types such as ``integers between $-2^{31}$ and $2^{31}-1$,'' ``32-bit
floating point numbers,'' ``strings of a certain maximum length,''
``arbitrary-precision decimals,'' and perhaps even ``geospatial coordinates.''

Does it matter? What about the column in table~\ref{tbl:penguins} labelled
``species''? To say that the values in this column are members of the domain
``strings'' seems like playing fast and loose with the world. A species is not a
string! Surely we ought to be allowed a domain, ``species of penguin''? It is
all very unclear.%
\footnote{And surely (surely!) domains are also relations?}

For now, though, ignore all this and assume that there is given, once and for
all, a set, $\mathscr{D} = \{\mathscr{D}_1, \mathscr{D}_2, \dotsc,
\mathscr{D}_N$, of domains $\mathscr{D}_i$.

We are also going to need some way to refer to the individual elements of a
tuple. Since an element of a mathematical relation is an ordered sequence, the
natural way to identify an element would seem to be by position and, indeed,
some treatments of the relational algebra do just this. However, for certain
parts of the theory this approach becomes cumbersome and we use an alternative
in which the elements are named.%
\footnote{Indeed, the positional notation is \emph{so} awkward that I suspect
that something deeper is going on with the named approach than merely notational
convenience. For example, \textcite{brown2022tables} note that the notion of
``pivoting'' a dataset requires ``first-class column names.'' In other words,
the column names themselves are allowed to become values (and \textit{vice
  versa}).}

Fix, once and for all, a countable set, $\mathscr{N}$, of \emph{attribute
names}, together with a map $D:\mathscr{N}\to\mathscr{D}$, which assigns, to each
attribute name, $n\in\mathscr{N}$, a domain, $D(n)\in\mathscr{D}$.\footnote{The
usual story doesn't bother to decompose the names by domain in this way, but
then has to add other caveats in subsequent definitions. Date, however, says
that (a) domains are types; and (b) a type is a \emph{named} set of values.} The
map $D$ should assign countably many names to each domain (so that there
``sufficiently many to work with.'')

A \emph{relation schema} is a finite set of attribute names. For $\mathcal{S}$ a relation
schema, a \emph{tuple over $\mathcal{S}$} is an assignment, to each attribute name $s\in\mathcal{S}$,
of an element of the domain $D(s)$. The elements of the tuple, which are pairs
of attribute name and value, are called \emph{attributes}. We write such an
element as $\mathrm{n}\colon v$ where $\mathrm{n}$ is a name and $v$ is an
element of the corresponding domain. We write a tuple using the notation
$\langle\mathrm{n}_1\colon v_1, \dotsc, \mathrm{n}_n\colon v_n \rangle$. The value of $n$,
the size of $\mathcal{S}$, is called the \emph{arity} of the relation.%
\footnote{This notation is not standard but I haven't found a standard one. What
\textcite{abiteboul1995foundations} do is to switch, \textit{ad libitum}, from
the named approach to the ordered sequence approach, depending on which one
provides a convenient notation at the time.}

A \emph{relation instance} (or sometimes just \emph{relation}) is a relation
schema, $\mathcal{S}$, together with a finite set of tuples over $\mathcal{S}$. The set of tuples is
sometimes called the \emph{extension} of $R$. For $R$ a relation, we write the
schema as $\schema(R)$ and the extension as $\extension(R)$.

A relation instance in this sense is essentially a relation in the mathematical
sense (except that a relation my have infinite extension but a relation instance
is by definition finite). Fix some ordering of the relation schema, say $(s_1,
s_2, \dotsc, s_n)$. Then each tuple is, by definition, an element of the
Cartesian product $D(s_1)\times D(s_2) \times\dotsb \times D(s_n)$. 

A database is composed of relation instances and it turns out we need names for
these as well (in part because an instance is not individuated by its schema nad
extension). Fix, once and for all, a set of \emph{relation names}. A
\emph{database instance} is a finite set of named relation instances.

In the ``RDBMS'' version of all this, a relation instance is called a
\emph{table}, the schema (or something like it) is called the \emph{header} of
the table, and a tuple is a \emph{row}.

The ensuing story raises at least four questions. What is a table
supposed to mean? What calculations can we effect on tables? What enquiries
might we make of collections of tables? And what other information is supported
by the relational model other than the brute existence of tables? We start with
the second question.

\subsection{The relational calculus}




Naturally, there are two of these. Why is nothing straightforward in this
subject? The two calculi are the \emph{tuple relational calculus} \parencite[][the
  original one]{codd1972relational} and the \emph{domain relational calculus}
\parencite{lacroix1977calculus}.%
\footnote{The abstract begins, ``A view of the semantics of a relational
database consists in considering that the database domains represent the sets of
objects of the subject matter and that the relations represent various kinds of
associations among these objects. This view is supported by query languages
where each variable ranges on a domain of the relational data base and
predicates correspond to the associations modeled by relations.''}

The broad principles of both calculi are the same, however. In both, one
specifies a relation by giving a logical expression, constructed using the usual
logical connectives together with quantification. This expression is known as a
``query.'' The extension of this query is the set of all tuples which make the
logical expression true. An implementation of the calculus is supposed to return
the extension as a relation instance. Although a general relation may have
infinite extension, certain restrictions on the allowable queries guarantee
finiteness.






\subsection{The relational algebra}

Table~\ref{tbl:relational-algebra} summarises the operations on relations known
collectively as the \emph{relational algebra.} These operations are of two broad
kinds: set-theoretic operations (union, intersection, and set difference) and
``relational'' operations (selection, projection, and join). Both kinds consume
and produce relations. There are also two constant relations and a way of
getting a relation given a tuple.

\begin{table}[ht]
  \begin{center}
    \begin{tabular}{@{}cl@{}}
      \toprule
      \multicolumn{2}{@{}l}{\textbf{Set-theoretic operations}} \\
      $R \cup S$ & Union  \\
      $R \cap S$ & Intersection \\
      $R \backslash S$ & Set difference \\
      \midrule
      \multicolumn{2}{@{}l}{\textbf{Relational operations}} \\
      $\pi_{\kappa} R$ & Projection onto schema $\kappa$ \\
      $\sigma_P R$ & Selection according to proposition $P$ \\
      $\rho_\pi R$ & Rename columns \\
      $R \times S$ & Cartesian product \\
      $R \bowtie S$ & Natural join \\
      \midrule
      \multicolumn{2}{@{}l}{\textbf{Constants and constructors}} \\
      $\dee$ & The nullary relation of one tuple \\
      $\dum$ & The empty nullary relation \\
      $\tau(\langle v_1, \dotsc, v_n\rangle)$ & A relation of one tuple \\ 
      \bottomrule
    \end{tabular}\label{tbl:relational-algebra}
    \caption{The relational algebra. $R$ and $S$ are arbitrary relations. Note
      that the set-theoretic operations require the operands to have the same
      schema. The above collection is redundant; a minimal set of operations
      (excluding constants) is $\cup$, $\setminus$,
      $\pi_\kappa$, $\sigma_P$, $\rho_\pi$, and $\bowtie$. $\dee$ and $\dum$ are not usually supported
      by SQL-based systems.}
  \end{center}
\end{table}

\subsubsection{Set-theoretic operations}

The binary operations $\cup$, $\cap$, and $\setminus$ are just what you would expect, with the
caveat that the two operands must have the same relation schema, which is also
the schema of the result.%
\footnote{When the definition of $\mathscr{N}$, the set of attribute names, is
simply a set, without the map to domains, this condition is trickier to state,
since one has to check equality of domains in addition to checking equality of
names.}

\subsubsection{Relational operations}

Let $R$ be a relation and suppose $\kappa$ is a subset of $\schema(R)$. The
\emph{projection}, $\pi_\kappa(R)$, is that relation, of schema $\kappa$, consisting of
tuples formed by omitting, from tuples in $R$, attributes whose name is not in
$\kappa$. Note that $\kappa$ may be empty and in that case $\pi_\kappa(R)$ is $\dum$ when $R$ is
empty and $\dee$ otherwise.

A \emph{propositional formula} over $\schema(R) = (n_1, \dotsc, n_n)$ is a
truth-valued function on $R$ of a particular form, described shortly. Given a
propositional formula, $P$, a \emph{selection}, $\sigma_P(R)$, is that relation
consisting of all tuples $x\in R$ such that $P(x)$ is true.

The particular forms that $P$ can take are those that are formed from
\emph{propositional atoms} by the logical connectives $\wedge$, $\vee$, and $\neg$. What
counts as a propositional atom seems to vary. At a minimum, they are expressions
of the form $x_i = \mathrm{a}$ or $x_i = x_j$. Here, $x_i$ and $x_j$ are
elements of a tuple in $R$ having the same domain, and `$\mathrm{a}$' is a
constant from the same domain as $x_i$. These atoms allow one to ``filter on
specific values of specific columns.''

However, sometimes other atoms are allowed, for example, the expression $x_1 >
0$, or $x_1 \leq x_2$. That is, a given system might allow relations other than
equality. (Which anyway seems special.) Once one admits \emph{anything} more than
equality, why not \emph{any} binary relation? Indeed, why not any relation? In
which case, why are we expressing what is clearly a relational concept in this
form?

I think what's going on is that we're trying to keep everything finite. The
domains are allowed to be (countably) infinite, so the extension of a general
binary relation such as $<$ is also an infinite set; and for reasons of
tractability we want to exclude infinite sets, or at least to confine the
infinities to the domains. So some things get coded as ``intensional'' and some
things---finite things---as ``extensional.''

For $R$ and $S$ any two relations whose schemas have empty intersection, the
\emph{Cartesian product} of $R$ and $S$ is the relation whose tuples are the
concatenation of a tuple in $R$ with a tuple in $S$. The schema of $R\times S$ is
$\schema(R)\cup\schema(S)$.

Finally, let $R$ and $S$ be relations whose schemas may have common
attributes. The \emph{natural join,} $R \bowtie S$, is the relation of schema
$\schema(R) \cup \schema(S)$ consisting of all pairs of tuples from $R$ and $S$
which agree on their common attributes.%
\footnote{Sometimes one introducts a ``$\theta$-join'', equivalent to taking the
Cartesian product and then selecting based on some propositional formula $\theta$. If
all the atoms in $\theta$ are equalities, the result is called an ``equi-join.''}

Note that if $\schema(R)\cap\schema(S)=\emptyset$ then the natural join is the Cartesian
product; and if $\schema(R) = \schema(S)$ then the natural join is the
intersection. Thus one can get away without $\times$ or $\cap$ as long as one is free to
\emph{rename} attributes. Let $\pi:\schema(R)\to\mathscr{N}$ be an injective map
preserving domains. (That is, for any attribute name $s\in\schema(R)$, the domain
of $\pi(s)$ is the same as the domain of~$s$.) Then $\rho_\pi R$ is that relation,
whose schema is the image of $\pi$, obtained by renaming the attributes in each
tuple as indicated by~$\pi$.

\subsubsection{Meaning}
What does a tuple mean? On Codd's view, a tuple is a statement of a fact: that
is, it is a proposition. Specifically, it is the proposition that some
predicate, when applied to the values in the tuple, is true.\footnote{That's why
the set of tuples is a set. Date reports Codd as saying, ``If something is true,
saying it twice doesn’t make it any more true.''} What predicate does a relation
express?

Here things are murky again. A logical predicate is usually interpreted as
saying that some relationship holds between individuals. For example, $1<2$
means ``a relationship of `less than' holds between the individual 1 and the
individual 2.'' But, on the face of it, table~\ref{tbl:penguins} doesn't seem to
be of that form. Imagine projecting this relation onto the attributes
``species'' and ``island''. One tuple in that resulting relation is $\langle
\mathtt{Gentoo}, \mathtt{Biscoe} \rangle$. What relation holds between the species
``Gentoo'' and the island ``Biscoe''? In this example, the relation must express
something like ``\emph{there exists} a penguin, such that this penguin is a
Gentoo and lives on Biscoe.'' That is, this relation asserts the existence of
some individual (a penguin); moreoever---and this is definitely confusing---this
individual is \emph{not} a value of one of the domains.

\subsubsection{What's missing?}

\begin{itemize}
\item In the algebra described above, there is no ``grouping'' or aggregation.
\item Pivot and unpivot
\item Transitive closure
\end{itemize}




\section{Sets and logic}




intension vs extension












\printbibliography

\end{document}
