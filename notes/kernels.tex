% Created 2022-03-24 Thu 12:27
% Intended LaTeX compiler: xelatex
\documentclass[11pt]{article}
\usepackage{graphicx}
\usepackage{grffile}
\usepackage{longtable}
\usepackage{wrapfig}
\usepackage{rotating}
\usepackage[normalem]{ulem}
\usepackage{amsmath}
\usepackage{textcomp}
\usepackage{amssymb}
\usepackage{capt-of}
\usepackage{hyperref}
\usepackage{fontspec}
\usepackage{newfloat}
\usepackage{minted}
\usepackage{booktabs}
\author{James Geddes}
\date{\today}
\title{The representer theorem and kernels: Notes}
\hypersetup{
 pdfauthor={James Geddes},
 pdftitle={The representer theorem and kernels: Notes},
 pdfkeywords={},
 pdfsubject={},
 pdfcreator={Emacs 27.2 (Org mode 9.5.2)}, 
 pdflang={English}}
\begin{document}

\maketitle

\section{Introduction: Linear Regression}
\label{sec:orgd50aef0}

In a regression problem there is given empirical data which we are to model. The
data consist of pairs \((x_1, y_1)\), \((x_2, y_2)\), \ldots{}, \((x_n, y_n)\) and we want
to find a function \(\tilde{f}(x)\) that `approximates' this data in the sense
that \(\tilde{f}(x_i)\) are close to the \(y_i\). The data themselves are supposed
to come from some `true' function, possibly with a bit of randomness on top.

A common way to define `close to' is by saying that \(f\) approximates the data if
the value of
\begin{equation*} 
E[f\mid x_i, y_i] = \sum_i \bigl(y_i - f(x_i)\bigr)^2
\end{equation*} 
is small. 

We call \(E\) the `error functional'. It's a `functional' because its argument is a
function.\footnote{Although of course functions whose domain is a space of function are
still functions.}

Well, it's rather \emph{too} easy to find a function which \emph{exactly} matches the
data: consider the function that is zero everywhere, except at the \(x_i\) where it
is equal to the \(y_i\). The problem with this function is that it's very unlikely
to give the right answer on new data. So what we really want is a function which
`smoothly interpolates,' or `is useful for prediction'.

Thus, in practice, we first pick a family of `reasonable' functions,
\(\mathcal{F}\), and we restrict ourselves to functions \(f\in\mathcal{F}\) in that
family.

For example, in simple linear regression, the family \(\mathcal{F}\) might be
taken to be `all straight lines.' A function in this family is parameterised by
two numbers, \(\alpha_0\) and \(\alpha_1\), and is of the form \(f(x) = \alpha_0 +
\alpha_1 x\).

Often, however, the data clearly do not lie on a straight lines. In these cases,
we might expand the family \(\mathcal{F}\). For example, we might add quadratic
terms and consider all functions of the form \(f(x) = \alpha_0 + \alpha_1 x +
\alpha_2 x^2\). Now we can fit slightly curvy data.

Note that there is something special about this particular family; namely, that
it has the structure of a vector space. For suppose \(f\) and \(g\) are of this
form, so that \(f(x) = \alpha_0 + \alpha_1 x + \alpha_2 x^2\) and \(g(x) =
\beta_0 + \beta_1 x + \beta_2 x^2\). Then the function \(f(x) + c g(x)\), where \(c\)
is a number, is also in the same family. In other words, linear combinations of
functions in \(\mathcal{F}\) are in \(\mathcal{F}\). This is why it's called linear
regression: not because it only contains linear functions but because the family
itself is a `linear space.'

In summary, in ordinary linear regression, we are trying to solve this
\emph{optimisation} problem:
\begin{equation}
\tilde{f} = 
\operatorname*{arg\,min}_{f\in\mathcal{F}} E[f\mid x_i, y_i] =
\operatorname*{arg\,min}_{f\in\mathcal{F}} \sum_i \bigl(y_i - f(x_i)\bigr)^2.
\end{equation}


\subsection{Regularised regression}
\label{sec:orgb110553}

There's a tension to this game. If \(\mathcal{F}\) is `too small'---for example,
if it consists of just the straight lines---then even the best-fitting member of
this family, the one that minimises \(E\), is unlikely to be a good fit. On the
other hand, if \(\mathcal{F}\) is `too large'---for example, if it is all
functions---then many functions will match the data, and most of them are
extremely unlikely to be close to the `true' function.

A popular way to resolve this tension is to start with a capacious \(\mathcal{F}\)
but to include a preference for some functions over others. For example, we
might prefer less-wiggly functions over more-wiggly functions. Then the
optimisation becomes a matter of balancing the goodness-of-fit (of the function
to the data) with the degree to which we prefer a particular function.

Conceptually, we imagine there is some other functional, \(R[f]\), known as the
`regulariser', which measures the extent to which we don't like \(f\). Now we are
looking to solve the problem of minimising the sum of the error functional and
the regulariser, known as the `loss functional':
\begin{equation}
\tilde{f} = \operatorname*{arg\,min}_{f\in\mathcal{F}} L[f],
\end{equation}
where
\begin{equation} 
L[f] = E[f] + R[f].
\end{equation}

How might we choose \(R\)? In the case of linear regression---when \(\mathcal{F}\) is
a vector space---there are two common choices. Suppose that in some basis the
components of a vector are \(\alpha_i\) (just like in the examples above). One
choice of regulariser is
\begin{equation}
R[f] = \lambda \sum_p \alpha_p^2.
\end{equation}
This is called \emph{ridge regression} (or sometimes `Tikhonov
regularisation'). Another common choice is
\begin{equation}
R[f] = \lambda \sum_p \left|\alpha_p\right|,
\end{equation}
which is called \emph{lasso regression}. In both cases the \(\alpha_p\) are the
parameters of the regression function and \(\lambda\) is a term which controls the
balance between goodness-of-fit and function preference.


\section{A representer theorem}
\label{sec:orgf57b0de}

The regression problem is therefore something about the `minimisation of a
function on a vector space.' Here is a theorem about certain minimisation
problems of this kind.

Let \(\mathcal{H}\) be a Hilbert space, with inner product \(\langle -,-\rangle\)
and norm \(\lVert v\rVert^2 = \langle v, v\rangle\). Let there be given vectors
\(w_1, w_2, \dotsc, w_n\) (known as `the data') and consider the following problem:
\begin{equation}
\operatorname*{arg\,min}_{\phi\in\mathcal{H}} \Bigl( 
E(\langle\phi, w_1\rangle, \dotsc, \langle\phi, w_n\rangle) +
\lambda \lVert \phi \rVert^2
\Bigr). 
\end{equation}
Here \(E:\mathbb{R}^n\to\mathbb{R}\) is a function whose dependence on \(\phi\)
enters only though the inner products \(\langle \phi, w_i\rangle\); and
\(\lambda>0\) is some positive number. \emph{Theorem:} Any solution of this
minimisation problem is a linear combination of the \(w_i\).

The proof is surprisingly simple. The \(w_i\) span a subspace
\(\mathcal{W}\subset\mathcal{H}\) so that the Hilbert space can be decomposed into
the direct sum \(\mathcal{H} = \mathcal{W} \oplus\mathcal{W}^\perp\). Thus, any
vector \(\phi\) can be written (uniquely!) as \(\phi = w + z\) where \(w\in
\mathcal{W}\) and \(\langle z, w_i\rangle = 0\).)

Now consider the expression to be minimised. We have
\begin{equation}
\begin{aligned}
E(\langle\phi, w_1\rangle, \dotsc, \langle\phi, w_n\rangle) +
\lambda \lVert \phi \rVert^2 
&= 
E(\langle w, w_1\rangle, \dotsc, \langle w, w_n\rangle) +
\lambda \lVert w + z \rVert^2 \\
&\geq 
E(\langle w, w_1\rangle, \dotsc, \langle w, w_n\rangle) +
\lambda \lVert w \rVert^2.
 \end{aligned}
\end{equation}
The first line follows since \(\langle w + z, w_i\rangle = \langle w, w_i\rangle\),
and the second line follows since \(\lVert \phi \rVert^2 = \lVert w \rVert^2 +
\lVert z \rVert^2\).

Thus, if \(\tilde{\phi} = \tilde{w} + \tilde{z}\) is a minimiser, then necessarily
\(\tilde{z}=0\).

Incidentally, it's clear that one can replace the `regulariser' \(\lambda\lVert
v\rVert^2\) by \(\Omega(\lVert v\rVert)\) as long as \(\Omega\) is non-decreasing.

The importance of this result is that one might be able to replace minimisation
over very high-dimensional space (possibly infinite-dimensional) with a
minimisation problem over a lower-dimensional space (namely \(\mathcal{W}\)).

\section{Applying the representer theorem to regression}
\label{sec:orgcdca334}

Suppose \(\mathcal{F}\) is a Hilbert space with inner product \(\langle -,-
\rangle\), arising as a space of functions from some domain \(\mathcal{X}\) to
\(\mathbb{R}\) with addition and scalar multiplication defined pointwise.

Suppose we are given data \((x_i, y_i)\) and we wish to solve the regularised
regression problem
\begin{equation}
\tilde{\phi} = 
\operatorname*{arg\,min}_{\phi\in\mathcal{F}} \Bigl( E[\phi(x_i), y_i] + \lambda\lVert \phi\rVert^2 \Bigr).
\end{equation}
Here the error functional \(E\) depends on \(\phi\) only through its values at the
data, \(\phi(x_i)\).  

On the face of it, this does not look like the form required for the representer
theorem: the error functional depends on \(\phi\) though evaluation at the \(x_i\)
rather than by the inner product with some \(w_i\). However, consider the
`evaluation map,' \(\Delta_x\), defined by:
\begin{equation}
\begin{aligned}
\Delta_x : \mathcal{F} &\to \mathbb{R} \\
           \phi        &\mapsto \phi(x).
\end{aligned}
\end{equation}

By the definition of addition in \(\mathcal{F}\), the map \(\Delta_x\) is a linear
map. Suppose it is also a \emph{continuous} linear map (equivalently, a bounded
linear map\footnote{Boundedness means that, for all \(x\) there is some \(M\geq 0\) such
that \(\lVert \Delta_x \phi\rVert \leq M \lVert \phi\rVert\) for all \(\phi\).}). Then, by the Riesz representation theorem, there exists some
vector \(\delta_x \in\mathcal{F}\) such that
\begin{equation}
\Delta_{x} \phi = \langle \phi, \delta_{x}\rangle
\end{equation}
for all \(\phi\).



\section{The representer theorem for regression.}
\label{sec:org667e43a}

Here is the big observation. In regularised linear regression, the loss
functional usually depends on the function purely through the evaluation of the
function at the data, \(x_i\). If the vector space of functions under
consideration is such that \(\Delta_x\), the operation of `evaluation at \(x\),' is
a continuous linear map, then the premise of the representer theorem holds. In
this case we can conclude that the minimising function is a linear combination
of the \(\delta_{x_i}\).

Why is this helpful? It is helpful when the dimension of the vector space is
much larger than the number of data points, which is likely since this is
precisely when regularisation is useful. The dimension of the function space may
even be infinite. Under these circumstances the minimisation problem may be made
significantly more tractable.

In order to make practical use of this observation, it is necessary to be able
to find the vectors \(\delta_{x_i}\in\mathcal{F}\). Since elements of
\(\mathcal{F}\) are functions, we need to fund the functions \(\delta_{x_i}(x)\).

Consider simple linear regression, where the space \(\mathcal{F}\) consists of all
functions of the form \(f(x) = \alpha_0 + \alpha_1 x\). An obvious choice of
inner product is
\begin{equation*}
\langle \alpha_0 + \alpha_1 x, \beta_0 + \beta_1 x\rangle =
\alpha_0\beta_0 + \alpha_1\beta_1.
\end{equation*} 
Under this inner product, it's straightforward to see that
\begin{equation*}
\langle f(x), 1 + \tilde{x}x\rangle = \alpha_0 + \alpha_1 \tilde{x}, 
\end{equation*}
which is \(f(\tilde{x})\), \emph{i.e.}, the evaluation of \(f\) at \(\tilde{x}\). Thus we have
\begin{equation*}
\delta_{\tilde{x}}(x) = 1 + \tilde{x}x.
\end{equation*}
(Here, the right hand side is a member of \(\mathcal{F}\) in virtue of being a
function of \(x\).)

A Hilbert space of functions in which \(\Delta_x\) is a continuous linear map is
called a \emph{reproducing kernel Hilbert space}, abbreviated RKHS. The next section
explains why.

\section{Kernels}
\label{sec:org55e7ec8}

Suppose \((\mathcal{F}, \langle, -, -\rangle)\) is an RKHS, whose underlying
functions have domain \(\mathcal{X}\) and range \(\mathbb{R}\). By definition, for any \(x,
x'\in\mathcal{X}\) there are functions \(\delta_{x}, \delta_{x'}\in\mathcal{F}\)
representing the action of evaluation. Since, say, \(\delta_{x'}\) is a function,
we might ask for its functional form; that is, for \(\delta_{x'}(x)\). By
definition,
\begin{equation*}
\delta_{x'}(x) = \langle \delta_{x'}, \delta_{x}\rangle.
\end{equation*}
The function on the right hand side, 
\begin{equation*}
\begin{aligned}
K : \mathcal{X}\times\mathcal{X} & \to \mathbb{R} \\
     (x, x')                     & \mapsto \langle \delta_x, \delta_{x'}\rangle. 
\end{aligned}
\end{equation*}
is known as the \emph{kernel}.\footnote{The word `kernel' is severely overloaded in mathematics. A kernel is
typically a symmetric function of two variables although, as here, the
definition mostly seems to reflect how the function is used.} It is symmetric (becuase the inner product is
symmetric) and \emph{positive definite}. Positive definite means that, for any
numbers \(c_1, \dotsc, c_n\in\mathbb{R}\), and elements \(x_1, \dotsc, x_n\in
\mathcal{X}\),
\begin{equation*}
\sum_i \sum_j c_i c_j K(x_i, x_j) \geq 0.
\end{equation*}
with the inequality being saturated if and only if all the \(c_i\) are zero. The
kernel is positive definite since
\begin{equation*}
\sum_i \sum_j c_i c_j K(x_i, x_j) = \langle v, v\rangle \geq 0,
\end{equation*}
where \(v = \sum_i c_i\delta_{x_i}\) and the final inequality is a property of
inner products.

This \(K(x, x')\) is sometimes called the \emph{reproducing kernel} (and gives its name
to the space) because it `reproduces' itself, in the sense that
\begin{equation*}
K(x, x') = \langle K(x, \cdot), K(x', \cdot)\rangle .
\end{equation*}

So we can now answer the question raised by the representer theorem: The
solution to the regularised linear regression problem is known to be a linear
combination of \(K(x_i, \cdot)\), the kernel evaluated at the data.

What is perhaps more interesting, and a large part of the usefulness of kernel
methods, is the following theorem:

\emph{Theorem (Moore--Aronszajn):} Let \(K:\mathcal{X}\times\mathcal{X}\to\mathbb{R}\)
be a symmetric, positive definite function. Then there exists a unique
reproducing kernel Hilbert space of functions on \(\mathcal{X}\) for which \(K\) is
the reproducing kernel.

(Idea of proof: complete the span of \(\{K(x, \cdot) \mid x\in\mathcal{X}\}\) in the
inner product defined by the reproducing property.)

The idea here is that one can get hold of lots of interesting RKHS's---and
potentially solve lots of interesting regularised regression problems---by
suitable choice of kernels. 
\end{document}