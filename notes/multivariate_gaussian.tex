\documentclass[10pt, a4paper, twocolumn]{article}
\usepackage[T1]{fontenc}
\usepackage{lmodern}
\usepackage{amssymb}
\usepackage{amsmath}
\usepackage{microtype}
\usepackage[margin=0.51in]{geometry}
%%
%\setlength{\parindent}{1em}
\setlength{\parskip}{\smallskipamount}
%%
\newcommand{\R}{\mathbb{R}}
\newcommand{\N}{\mathcal{N}}
\newcommand{\defn}[1]{\textbf{#1}}
\newcommand{\eg}{\emph{Example:}\relax}
\DeclareMathOperator{\dimension}{dim}
\DeclareMathOperator{\id}{id}
\DeclareMathOperator{\image}{Im}
\DeclareMathOperator{\kernel}{Ker}
%%
%% Multivariate Gaussians
%% This note was written by James Geddes
%%
\title{The Multivariate Gaussian}
\begin{document}\maketitle
The normal (or Gaussian) distribution is a probability distribution on $\R$,
parameterised by $\mu\in\R$ (the “mean”) and $\sigma>0 \in\R$ (the “standard deviation”),
and defined by its probability density function,\footnote{Sometimes we allow
  $\sigma=0$, when the distribution is then $\delta(x-\mu)$.}
\begin{equation*}
  \N(\mu, \sigma) \sim \frac{1}{\sigma\sqrt{2\pi}} e^{-(x-\mu)^2/2\sigma^2}.
\end{equation*}

Fix a finite-dimensional vector space,~$V$ and let $\mu$ be a probability
distribution on $V$. For $w:V\to\R$ any element of the dual of~$V$ we obtain a
probability distribution, $\mu_*$, on $\R$ in the following way: the probability
measure of some set $I\in\R$ is given by $\mu_*(I) = \mu(w^{-1}(I))$.\footnote{We are
  ignoring all questions of which sets are measurable.} This probability
distribution is called the “push-forward” of $\mu$ by~$w$.

A \defn{multivariate Gaussian} is a probability distribution on $V$ whose
push-foward to $\R$ by any element of the dual is a Gaussian distribution.






\end{document}
