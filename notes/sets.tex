\documentclass[10pt, a4paper, twocolumn]{article}
\usepackage[T1]{fontenc}
\usepackage{lmodern}
\usepackage{amssymb}
\usepackage{amsmath}
\usepackage{microtype}
\usepackage[margin=0.51in]{geometry}
%%
\newcommand{\defn}[1]{\textbf{#1}}
%%
%% Terminology from set theory
%%
\begin{document}
Set theory contains much of the vocabulary of mathematics. There is very little
content but there are quite a few concepts and these concepts permeate the rest
of the subject.

A \defn{set} is a “collection of things.” The term “collection” is not
defined. The things are called the \defn{elements} of the set. There are two
important facts about sets: within a set, the elements are unique; and there is
no order to the elements.

The set with no elements is called the \defn{empty set} and written as
$\emptyset$. There is only one empty set.

To describe a concrete set, with given elements, surround the elements with
braces. Thus, $\{0, 1, 2\}$ is the set containing the numbers 0, 1, and~2. It is
the same set as $\{2,0,1\}$. The expression $\{1, 1\}$ does not describe a
set. To show that an element is a member of some set, use “$\in$”. For example,
$1\in\{0,1,2\}$. To describe a set containing those elements of another set, $X$,
which satisfy some predicate $P(x)$, write $\{x \in X \mid P(x)\}$. For example, $\{x
\in \mathbb{N} \mid \text{$x$ is prime}\}$ is the set of primes.\footnote{There is a
  question of where “things” come from in the first place. Quite a lot of
  mathematics can be constructed out of the empty set. For example, $\{\emptyset\}$ is a
  set containing one element and $\{\emptyset, \{\emptyset\}\}$ is a set containing two
  elements. One might go so far as to \emph{define} 0 to be $\emptyset$, 1 to be
  $\{\emptyset\}$, and so on.}

The \defn{union} of two sets, $X$ and $Y$, written $X \cup Y$, is the set of all
elements in either $X$ or $Y$. The \defn{intersection} of two sets, written $X \cap
Y$, is the set of all elements in both $X$ and $Y$. The \defn{difference} of two
sets, written $Y \setminus X$ (or sometimes $Y-X$) is the set of all elements that are
in $Y$ but not in $X$. Sometimes there is a set $U$ that is understood to be the
“set of all things presently under consideration.” In that case, the
\defn{complement} of $X$, written $X^c$ (or sometimes $\bar{X}$) is the set of
all things in $U$ but not in $X$; that is, $X^c \equiv U \setminus X$.

There are many identities involving these operations that follow from the
definitions. For example: $X\cap(A\cup B) = (X\cap A)\cup (X\cap B)$. 

Given two sets, $X$ and $Y$, a \defn{map}, $f : X \to Y$, is a rule that assigns,
to each element $x \in X$, an element $f(x) \in Y$.\footnote{The term “function” is
  sometimes used to mean “map”; sometimes “function” means a map where the
  codomain is a number; sometimes “map” means a certain kind of
  structure-preserving map.} The set $X$ is called the \defn{domain} and the set
$Y$ is called the \defn{codomain}. Maps are the principal way by which we
compare sets and much of the subject of set theory is really the study of maps.

For $X$ any set, there is a unique map $\emptyset \to X$. For any set $X$, the identity
map, $\operatorname{id}_X : X \to X$, maps every point to itself.

A map is said to be \defn{one-to-one} (or \defn{injective}) if every point in
the domain is mapped to a unique point in the codomain. A map is said to be
\defn{onto} (or \defn{surjective}) if every point in the codomain is mapped to
from some point in the domain. Let $f:X \to Y$ be a map. The \defn{image} of $f$,
written $f(X)$, is the set $\{y \in Y \mid \text{$y = f(x)$ for some $x$}\}$. For any
$y \in f(X)$ we define $f^{-1}(y)$ as the set of all $x \in X$ that are mapped to
$y$ by~$f$. That is, $f^{-1}(y) = \{ x\in X\mid f(x) = y\}$.

if $f^{-a}$ has to “do something” with \emph{every} element of
the domain.


We sometimes do want to talk about collections of things where the order
matters. A \defn{pair} (sometimes “ordered pair”) is a collection of two things
where the order of the things matters and the two things might be the same
thing. For two things $x$ and $y$, a pair is written $(x, y)$. We call $x$ the
“first element” and $y$ the “second element”. The main property of pairs is that
$(x,y)=(p,q)$ if and only if $x=p$ and $y=q$.

One might wonder whether a pair is a new kind of thing. In fact, pairs can be
defined in terms of sets. Here is one way to do so: Suppose we happen to have
distinct things 1 and~2. Then we may define $(x, y) \equiv \{\{1, x\}, \{2,
y\}\}$. (This construction is called a “tagged union.”)

Suppose $X$ and $Y$ are sets. The \defn{Cartesian product} 


A \defn{finite tuple} is a pair where the second element may also be a pair. A
tuple $(a, (b, (c, d)))$ is written $(a, b, c, d)$. 









\end{document}
