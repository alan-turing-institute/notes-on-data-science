\documentclass[10pt, a4paper, twocolumn]{article}
\usepackage[T1]{fontenc}
\usepackage{lmodern}
\usepackage{amssymb}
\usepackage{amsmath}
\usepackage{microtype}
\usepackage[margin=0.51in]{geometry}
%%
%\setlength{\parindent}{1em}
\setlength{\parskip}{\smallskipamount}
%%
\newcommand{\defn}[1]{\textbf{#1}}
\newcommand{\eg}{\emph{Example:}\relax}
\DeclareMathOperator{\id}{id}
%%
%% Terminology from set theory
%%
\begin{document}
Set theory contains much of the vocabulary of mathematics. There is very little
content but there are quite a few concepts and these concepts permeate the rest
of the subject.

A \defn{set} is a “collection of things.” The term “collection” is not
defined. The things are called the \defn{elements} of the set. There are two
important facts about sets: within a set, the elements are unique; and there is
no order to the elements. To show that an element, $x$, is a member of some set,
$A$, write “$x \in A$”.

To describe a concrete set, with given elements, surround the elements with
braces. To describe a set containing those elements of another set, $X$, which
satisfy some predicate $P(x)$, write $\{x \in X \mid P(x)\}$. The set with no
elements is called the \defn{empty set} and written as $\emptyset$. There is only one
empty set.

\eg\ $\{0, 1, 2\}$ is the set containing the numbers 0, 1, and~2. It is the same
set as $\{2,0,1\}$.

\eg\ The expression $\{1, 1\}$ does not describe a set.

\eg\ $1\in\{0,1,2\}$.

\eg\ $\{x \in \mathbb{N} \mid \text{$x$ is prime}\}$ is the set of
primes.\footnote{There is a question of where “things” come from in the first place. Quite a lot of
  mathematics can be constructed out of the empty set. For example, $\{\emptyset\}$ is a set containing one
  element and $\{\emptyset, \{\emptyset\}\}$ is a set containing two elements. One might go so far as to \emph{define}
  0 to be $\emptyset$, 1 to be $\{\emptyset\}$, and so on.}

If every element of $B$ is also an element of $A$, we write $B \subset A$ and say that
$B$ is a \defn{subset} of~$A$. The set of all subsets of $B$ is called the
\defn{power set} of $B$ and written $\mathcal{P}(B)$.

\eg\ $\emptyset \subset B$ for any $B$. $\emptyset \in \mathcal{P}(B)$ for any $B$.

The \defn{union} of two sets, $X$ and $Y$, written $X \cup Y$, is the set of all
elements in either $X$ or $Y$. The \defn{intersection} of two sets, written $X \cap
Y$, is the set of all elements in both $X$ and $Y$. The \defn{difference} of two
sets, written $Y \setminus X$ (or sometimes $Y-X$) is the set of all elements that are
in $Y$ but not in $X$. Sometimes there is a set $U$ that is understood to be the
“set of all things presently under consideration.” In that case, the
\defn{complement} of $X$, written $X^c$ is the set of all things in $U$ but not
in $X$; that is, $X^c \equiv U \setminus X$. There are many identities involving these
operations that follow from the definitions.

\eg\ $X\cap(A\cup B) = (X\cap A)\cup (X\cap B)$.

Given two sets, $X$ and $Y$, a \defn{map}, $f : X \to Y$, is a rule that assigns,
to each element $x \in X$, an element $f(x) \in Y$.\footnote{The term “function” is
  sometimes used to mean “map”; sometimes “function” means a map where the
  codomain is a number; sometimes “map” means a certain kind of
  structure-preserving map.} The set $X$ is called the \defn{domain} and the set
$Y$ is called the \defn{codomain}. Maps are the principal way by which we
compare sets and much of the subject of set theory is really the study of maps.

\eg\ For $X$ any set, there is a unique map $\emptyset \to X$.

\eg\ For any set $X$, the
identity map, $\id_X : X \to X$, is the map $f(x) = x$ for every
$x\in X$. Where the set $X$ is implicitly known, we write simply “$\id$.” 

A map is said to be \defn{one-to-one} (or \defn{injective}) if every point in
the domain is mapped to a unique point in the codomain. The \defn{image} of $f$,
written $f(X)$, is the set $f(X) = \{y \in Y \mid \text{$y = f(x)$ for some
  $x$}\}$. If the image of $f$ is the codomain, the map is said to be \defn{onto} (or \defn{surjective}). 

For two maps $f:B \to C$ and $g:A \to B$, their \defn{composition}, $f\circ g: A \to C$
(read “$f$ after $g$”) is the map defined by $(f\circ g)(x) = f(g(x))$.

\eg\ $f\circ\id = \id\circ f = f$ for any map $f$. 

If, given $f:X \to Y$, there exists a $g:Y \to X$ such that $f\circ g = \id_Y$ and $g\circ f
= \id_X$, then we say that $f$ is an \defn{isomorphism} and that $X$ and $Y$ are
isomorphic. We often write $g = f^{-1}$.

\eg\ A map $f$ is an isomorphism if and only if it is both one-to-one and onto. 

\eg\ (\defn{Cantor's theorem}) There is no isomorphism between $A$ and $\mathcal{P}(A)$. 

We sometimes do want to talk about collections of things where the order
matters. A \defn{pair} (sometimes “ordered pair”) is a collection of two things
where the order of the things matters and the two things might be the same
thing. For two things $x$ and $y$, a pair is written $(x, y)$. We call $x$ the
“first element” and $y$ the “second element”. The main property of pairs is that
$(x,y)=(p,q)$ if and only if $x=p$ and $y=q$.

One might wonder whether a pair is a new kind of thing. In fact, pairs can be
defined in terms of sets. Here is one way to do so: $(x, y)\equiv \{x, \{x,y\}\}$.

A \defn{finite tuple} is an ordered list of things, possible repeated. One may
define a 3-tuple, $(a, b, c)$ say, as a pair whose second element is a pair:
$(a, (b, c))$. 

Suppose $X$ and $Y$ are sets. Their \defn{Cartesian product}, $X \times Y$, is the
set of all pairs $(x, y)$ where $x \in X$ and $y \in Y$. The Cartesian product comes
with two natural \defn{projection} maps; namely, $\pi_X : X \times Y \to X$ and $\pi_Y : X
\times Y \to Y$.

\eg\ $(\pi_X p, \pi_Y p) = p$ for any $p \in X \times Y$.

The Cartesian product is associative, in the sense that there is a
``natural'' isomporphism between $X \times (Y \times Z)$ and $(X
\times Y) \times Z$. Because of this, we normally write simply $X
\times Y \times Z$.

\end{document}
