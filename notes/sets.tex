\documentclass[12pt, a4paper]{article}
\usepackage[T1]{fontenc}
\usepackage{beton}
\usepackage{eulervm}
\DeclareFontSeriesDefault[rm]{bf}{sbc}
\usepackage[font=footnotesize]{caption}
\usepackage{ragged2e}
\usepackage{snotez}
\setsidenotes{
  text-format+=\RaggedRight}
\usepackage{amsmath}
\usepackage{microtype}
\usepackage[a4paper]{geometry}
\geometry{left=15mm,
  textwidth=0.6\paperwidth, % 126mm for A4
  textheight=0.6\paperheight}
%\usepackage{calc}
\geometry{marginparsep=5mm, marginparwidth=49mm}
\geometry{footskip=2in}
%%
\usepackage{tikz-cd}
%%
\setlength{\parskip}{\smallskipamount}
%%
\newcommand{\defn}[1]{\textbf{#1}}
\newcommand{\set}[1]{\mathbold{#1}}
\newcommand{\eg}{\emph{Example:}\relax}
\newcommand{\id}{\text{id}}
%\DeclareMathOperator{\id}{id}
%%
%%
\title{\vspace{-6ex}Sets}
\author{James Geddes}
\begin{document}
\maketitle

Set theory contains much of the vocabulary of mathematics. This note
is a short summary of those parts of set theory which you are likely
to meet when studying other topics. There is not much here beyond some
concepts and examples but these concepts permeate the rest of
mathematics: most mathematical objects you will meet will be “sets,
with some additional structure,” and in many cases the additional
structures are themselves sets.

A \defn{set} is a “collection of things.” The term “collection” is not
defined. The things are called the \defn{elements} (or \defn{members})
of the set. To show that a thing, $x$, is a member of some set,
$A$, write $x \in A$. A set is nothing more than the collection of its
elements: two sets are equal if and only if they have the same elements.

There are two important facts about sets: within a set, the elements
are unique; and there is no order to the elements.

To describe a concrete set with given elements, surround the elements
with braces. To describe a set containing those elements of another
set, $X$, which satisfy some predicate $P(x)$, write $\{x \in X \mid
P(x)\}$. The set with no elements is called the \defn{empty set} and
written as $\emptyset$. There is only one empty set.

\eg\ $\{0, 1, 2\}$ is the set containing 0, 1, and~2. It is the same
set as $\{2,0,1\}$.

\eg\ The expression $\{1, 1\}$ does not describe a set.

\eg\ $1\in\{0,1,2\}$. $3\notin\{0,1,2\}$.

\eg\ The symbol $\set{N}$ denotes the set of natural numbers,
$\{0,1,2,\dotsc\}$. The set $\{n \in \set{N} \mid \text{$n$ is prime}\}$ is
the set of primes.

\eg\ The set $\{\emptyset\}$ has one element (the empty set). The set $\{\emptyset,
\{\emptyset\}\}$ has two elements (the empty set, and the set containing the
empty set).

If every element of $B$ is also an element of $A$, we write $B \subset A$
and say that $B$ is a \defn{subset} of~$A$. The set of all subsets of
$A$ is called the \defn{power set} of $A$ and written $\mathcal{P}(A)$.

\eg\ The set of all even natural numbers is a subset of the set of all
natural numbers: $\{2n \mid n \in \set{N}\} \subset \set{N}$.

\eg\ The power set of $\{0,1,2\}$ has, as elements, $\emptyset$, $\{0\}$,
$\{1\}$, $\{2\}$, $\{0,1\}$, $\{0,2\}$, $\{1,2\}$, and $\{0,1,2\}$.

\eg\ $\emptyset \subset A$ for any $A$. Therefore, $\emptyset \in \mathcal{P}(A)$ for any $A$. 

The \defn{union} of two sets $X$ and $Y$, written $X \cup Y$, is the set
of all elements of either $X$ or $Y$. The \defn{intersection} of two
sets, written $X \cap Y$, is the set of all elements in both $X$ and
$Y$. The \defn{set difference} of two sets, written $Y \setminus X$ (or
sometimes $Y-X$) is the set of all elements that are in $Y$ but not in
$X$.

Sometimes there is a set $\mathcal{U}$ that is understood to be the “set of all
things presently under consideration.” In that case, the
\defn{complement} of $X$, written $X^c$, is the set of all things in
$\mathcal{U}$ but not in $X$; that is, $X^c \equiv \mathcal{U} \setminus X$.

There are many identities involving these operations that follow from
the definitions.
\eg\ $X\cap(A\cup B) = (X\cap A) \cup (X\cap B)$.

We sometimes do want to talk about collections of things where the
order matters. A \defn{pair} (sometimes “ordered pair”) is a
collection of two things where the order of the things matters and the
two things might be the same thing. For two things $x$ and $y$, a pair
is written $(x, y)$. We call $x$ the “first element” and $y$ the
“second element.” The defining property of pairs is that $(x,y)=(p,q)$
if and only if $x=p$ and $y=q$. One might wonder whether a pair is a
new kind of thing. In fact, pairs can be defined in terms of
sets. Here is one way to do so: $(x, y)\equiv \{\{x\}, \{x,y\}\}$. (Though
it's surprisingly subtle to say why this works.)

An \defn{$n$-tuple} (or just \defn{tuple}) is an ordered list of $n$
things. A 3-tuple, say, is written $(x, y, z)$. (Note that any of $x$,
$y$, and $z$ might be the same thing.) One way of defining a 3-tuple,
$(a, b, c)$, is as a pair whose second element is a pair: $(a, (b,
c))$.
  
Tuples provide one way of constructing a new set out of two given
sets. Let $X$ and $Y$ be sets. Their \defn{Cartesian product} (or
simply \defn{product}), $X \times Y$, is the set of all pairs $(x, y)$
where $x \in X$ and $y \in Y$. That is,
\begin{equation*}
X\times Y\equiv \{(x, y) \mid x\in X\;\text{and}\; y\in Y\}.
\end{equation*}

\eg\ $\set{R}^2$ is the set of all pairs of real numbers. If you think
of $\set{R}$ as ``the real line,'' then you can think of $\set{R}^2$
as ``the plane.''

\eg\ $\{0,1\}\times\{a, b\} = \{(0,a), (0,b), (1,a), (1,b)\}$. 

The product of two sets is “all combinations of elements, one from
each set.” There is a related construction which provides “all the
elements of one set, together with the elements of another set, where,
unlike the set union, we keep track of which set the elements came
from.” Let $X$ and $Y$ be sets. Their \defn{coproduct} (or sometimes,
\defn{disjoint union}), $X \sqcup Y$, is the set
\begin{equation*}
  X \sqcup Y \equiv \{(0,x) \mid x\in X\} \cup \{(1, y)\mid y\in Y\}.
\end{equation*}
The coproduct contains all the elements of $X$ “tagged” with $0$,
together with all the elements of $Y$ “tagged” with $1$. If some
element $\text{a}\in X$ is also an element of $Y$, it will “appear twice” in the
coproduct, once as $(0,\text{a})$ and once as $(1,\text{a})$. 

The above is largely all that we shall say about the subject of sets
as sets. There are a few definitions and some results which follow
more or less immediately from the definitions. A richer view is
provided by the relationships between sets, perhaps the most important
of which is the notion of a map.

\begin{sidefigure}
  \[\begin{tikzcd}
    X \arrow[r, "f"] & Y
  \end{tikzcd}\]
  \caption{A map, $f\colon X\to Y$, with domain $X$ and codomain $Y$.}
\end{sidefigure}
Given two sets, $X$ and $Y$, a \defn{map}, $f\colon X \to Y$, is a rule
that assigns, to each element $x \in X$, an element $f(x) \in
Y$.\footnote{The term “function” is sometimes used instead of “map;”
sometimes “function” means a map where the codomain is a number;
sometimes “map” means a certain kind of structure-preserving map.} The
set $X$ is called the \defn{domain} of $f$ and the set $Y$ is called
the \defn{codomain} of~$f$. The rule must cover \emph{every} element
of the domain. Much of any given topic in mathematics is the study of
the maps relevant to that topic.

A map may be described using the notation $f\colon x\mapsto f(x)$. (The
symbol $\mapsto$ is read as ``maps to.'')

\eg\ For $X$ any set, there is a unique map $\emptyset \to X$. (It may surprise
you that there is such a map at all. After all, there are no elements
in $\emptyset$ to be mapped to something in $X$. However, a map just has to
say what to do with each element of its domain and, since this domain
is empty, the “empty set of rules” is a perfectly fine map.)

\eg\ For any set $X$, the \emph{identity map}, $\id_X\colon X \to X$, is
the map given by $\id_X\colon x \mapsto x$ for every $x\in X$. Where the set
$X$ is implicitly known, we write simply “$\id$.”
\begin{sidefigure}
\[\begin{tikzcd}
    X \arrow[loop right, out=45, in=-45, distance=3em, "\id"]
  \end{tikzcd}\]
  \caption{The identity map $\id\colon X\to X$.}
\end{sidefigure}

The main operation on maps is that of composition. Suppose there are
two maps, $f\colon A\to B$ and $g\colon B\to C$, as shown in
figure~\ref{fig:comp}. These two maps are \emph{compatible} in the
sense that the codomain of $f$ is the domain of~$g$.
\begin{sidefigure}
  \[
  \begin{tikzcd}
    A \arrow [rr, bend right, "g \circ f"' ] \arrow[r, "f"] & B \arrow[r, "g"] & C 
  \end{tikzcd}
  \]
  \caption{Composition of maps, $g \circ f$. Notice that $g$ is applied
    after $f$.\label{fig:comp}}
\end{sidefigure}
For compatible maps, one may construct a map $A\to C$ by the following
rule: start with an element $x\in A$, then use $f$ to obtain an element
$f(x)\in B$, then use $g$ to obtain an element $g(f(x))\in C$. That is,
for two maps $f\colon A \to B$ and $g\colon B \to C$, their
\defn{composition}, $g\circ f\colon A \to C$, is the map defined by $(g\circ
f)(x) = g(f(x))$. Notice that $g\circ f$ means “apply $g$ \emph{after}
applying $f$,” and we normally read $g\circ f$ as “$g$ after
$f$.”\footnote{I picked up this way of reading $g\circ f$ from Bartosz
  Milewski's lectures on category theory. It helps me to remember
  which way round the maps are!}
  
\eg\ For any map $f\colon A\to B$, we have $f\circ\id_A = \id_B\circ f = f$.

\eg\ Composition of maps is associative: for any three compatible
maps, $h\circ (g\circ f) = (h\circ g)\circ f$.

There are a few special kinds of maps. 
 
A map is said to be \defn{injective} (or \defn{one-to-one}) if every
element of the domain is mapped to a \emph{unique} element in the
codomain. Injective maps “keep distinct elements of the domain
distinct.”

\begin{sidefigure}
\[\begin{tikzcd}
  X \arrow[r, bend left, "f"] \arrow [r, bend right, "g"] & A \arrow[r, "i"] & B
\end{tikzcd}\]
  \caption{An injective map, $i$.\label{fig:injection}}
\end{sidefigure}
Injective maps have the following property. Suppose $i\colon A\to B$ is
an injective map. For any set $X$ and maps $f\colon X\to A$ and $g\colon
X\to A$, if $i\circ f = i\circ g$ then $f = g$. (To see this, note that if $f$
and $g$ were different maps, they would differ on some $x \in X$; then
$i$ would take $f(x)$ and $g(x)$ to different points in $B$, otherwise
it would not be injective, and so the compositions could not be equal
either.) That is, injective maps are “maps that can be cancelled out
on the left.”

The \defn{image} of $f$, written $f[X]$, is the set of all elements of
the codomain mapped to from \emph{some} element of the domain: $f[X] =
\{f(x) \in Y \mid x\in X\}$. If the image of $f$ is the entire codomain, the
map is said to be \defn{surjective} (or \defn{onto}). Surjective maps
“don't miss out parts of the codomain.”

\begin{sidefigure}
\[\begin{tikzcd}
  A \arrow[r, "s"] & B \arrow[r, bend left, "f"] \arrow [r, bend right, "g"] & X 
\end{tikzcd}\]
  \caption{A surjective map, $s$.\label{fig:surjection}}
\end{sidefigure}
Surjective maps have the following property. If $s\colon A\to B$ is
surjective then, for any set $X$ and maps $f\colon B\to X$ and $g\colon
B\to X$, if $f\circ s = g\circ s$ then $f = g$. (To see this, note that if $f$
and $g$ differed on any $b\in B$, then, since there is some $a\in A$ with
$b = s(a)$, the compositions would differ on this $a$.) That is,
surjective maps are “maps that can be cancelled out on the right.”

\eg\ The identity map is both injective and surjective.

\eg\ The map $\set{R}\to\set{R}$ defined by $x\mapsto x^2$ is neither
injective nor surjective. It is not injective because, for example,
both $2$ and $-2$ are mapped to $4$. It is not surjective because the
codomain is only the non-negative reals.

There is a convenient, graphical, way to give the conditions for being
injective and surjective, illustrated in figures~\ref{fig:injection}
and~\ref{fig:surjection}. Consider each diagram as a directed graph,
with the sets forming the vertices and the maps forming the
edges. There may be certain pairs of sets for which multiple paths
exist between the two. For example, in figure~\ref{fig:injection} one
path from $X$ to $B$ is “first $f$ then $i$” and another path is
“first $g$ then $i$.” The premise in the definition of injective is
precisely that “it doesn't matter which path you take.”

We say that a diagram \defn{commutes} if every composition of maps
resulting in the same domain and codomain gives rise to the same map.

\eg\ A map $i\colon A\to B$ is injective if and only if it is such that,
whenever the diagram of figure~\ref{fig:injection} commutes, then
$f=g$.

\eg\ A map $s\colon A\to B$ is surjective if and only if it is such that,
whenever the diagram of figure~\ref{fig:surjection} commutes, then
$f=g$.

The \defn{inverse} of a map $f\colon A\to B$ is a map $g\colon B\to A$
such that $g\circ f = \id_A$ and $f\circ g = \id_B$ (see
figure~\ref{fig:inverse}). The inverse of $f$, if it exists, is
sometimes written as $f^{-1}$.
\begin{sidefigure}
  \[\begin{tikzcd}
  A \arrow[loop left, out=225, in=135, distance=3em, "\id"] \arrow[r, bend
      left, "f"] &
    B \arrow[l, bend left, "g"] \arrow[loop right, out=45, in=-45,
      distance=3em, "\id"]
  \end{tikzcd}\]
  \caption{An inverse of a map $f$ is a map $g$ which makes this
    diagram commute.\label{fig:inverse}}
\end{sidefigure}
A map may not have an inverse; in fact, an inverse of $f$ exists if
and only if $f$ is both injective and surjective. Consider trying to
find the inverse $f^{-1}(y)$ of some element $y\in B$. Because $f$ is
surjective, we know there is \emph{some} $x\in A$ such that $f(x) = y$;
and because it is injective, we know that this $x$ is unique. That
defines the inverse on each element of~$B$. A map with an inverse is
called an \defn{bijection} (or an \defn{isomorphism of
  sets}).\footnote{There seem to be two names for everything:
“injective” and “one-to-one;” “surjective” and “onto;” “one-to-one
correspondence” (which we haven't used) and “bijection.”  The terms
“one-to-one” (and perhaps “onto”) seem to be the oldest. It was
Bourbaki (a French mathematics collective!) who introduced
“injection,” “surjection,” and “bijection,” perhaps because they
wanted to make clear that the domain and codomain of a map are an
integral part of its specification.} To denote the existence a
bijection between sets $A$ and $B$ we write $A\cong B$. 

\eg\ A bijection exists between two finite sets (sets with finitely
many members) if and only if the two sets have the same number of
elements.

\eg\ (\defn{Cantor's theorem}) There is no bijection between $A$ and
$\mathcal{P}(A)$ for any set~$A$.

\eg\ A set is called \emph{countable} if there exists a bijection
between it and the natural numbers. The integers and the rationals are
both countable. If a set is not countable, it is said to be
\emph{uncountable}. The real numbers are uncountable. In this sense,
there are ``more'' real numbers than there are natural numbers.

The notion of countability gives rise to “counting arguments.” Here is
an example. A real number $r$, between 0 and 1, is said to be
\emph{computable} if there exists a computer program which, for any
natural number $n$, will output in finite time the $n$th digit in the
decimal expansion of~$r$. \emph{Claim:} Almost every number is not
computable. \emph{Proof:} There are uncountably many reals between 0
and~1 but there are only countably many programs. Hence most reals are
not computable. (There are only countably many programs because a
program is a finite list of instructions and there are only finitely
many possible instructions: it turns out that the set of all finite
lists of elements from a finite set is countable.)

In the introduction, we said that most mathematical objects are sets
and it turns out that this is true of maps. A map, $f:X\times Y$, is a kind
of relationship between two sets: it pairs each element, $x$, of its
domain with an element, $f(x)$, of its codomain. In other words, a map
can be represented by the set of pairs $\{(x,f(x)) \mid x\in X\}$. In
short, a map is a subset of $X\times Y$. Not every subset of $X\times Y$
represents a map: to represent a map, there must be one and only one
pair $(x,y)$ in the subset for every element $x\in X$. The subset of $X\times
Y$ generated by a map is sometimes called its
\emph{graph}.\footnote{This way of thinking is described as the
“extensional” view: on this view, the map simply \emph{is} the set
that comprises its graph, without regard to what it “means.” The
extensional view contrasts with our earlier, “intensional,” view, on
which a map is a rule. I don't fully understand this distinction so I
won't say more about it here.}

Finally, suppose $A$ and $B$ are sets, and consider the collection of
\emph{all} maps $A\to B$ (see figure~\ref{fig:homset}).
\begin{sidefigure}
  \[\begin{tikzcd}
  A
  \arrow[r, out=-60, in=-120, "f_1"]
  \arrow[r, out=10, in=170, "f_2"]
  \arrow[r, out=70, in=110, "f_3"] & B 
  \end{tikzcd}\]
  \caption{A selection of maps from $A$ to $B$.\label{fig:homset}}
\end{sidefigure}
This collection is itself a set, written as~$B^A$. It is sometimes
called the \emph{hom-set} (a term from category theory). The notation
is suggestive of exponentiation as the following examples illustrate.

\eg\ Denote by $\set{2}$ the set of two elements:
$\set{2}\equiv\{0,1\}$.\footnote{We say \emph{the} set because we will not
care what those elements are, just that there are two of them. If you
don't like using numbers (because where do they come from?) you can
use the set $\{\emptyset, \{\emptyset\}\}$.} Then, for $X$ any set, $X^\set{2}$ is the
set of all maps from $\set{2}$ to $X$. A map from $\set{2}$ to $X$ is
a rule that assigns an element of $X$ to each of the two elements
of~$X$; that is, it is a pair of elements of~$X$: in other words,
it is an element of $X\times X$. Likewise, every element of $X\times X$ is a
pair of elements of $X$ and thus gives rise to a map $\set{2}\to
X$. Thus, there exists a bijection, $X^\set{2} \cong X\times X$.

\eg\ $\set{2}^X \cong \mathcal{P}(X)$. 

As a final example, note that a map always has a single domain;
it “takes a single argument.” What if we want a map which “takes two
arguments” (a common feature of functions in a computer program)? One
way to do this is to package up the two arguments as a pair. For
example, a map $f:A\times B\to C$ can be thought of as a map which “takes an
$A$ and a $B$ and returns a~$C$.”

\eg\ $C^{A\times B} \cong (C^B)^A.$ That is, “a map from pairs of $A$ and $B$ to
$C$ is the same as a map which takes an $A$, and returns a map from
$B$ to $C$.” In computer science, passing from the left hand side to
the right hand side of this bijection is called “currying,” after
Haskell Curry.



\end{document}


The Cartesian product comes with two natural \defn{projection maps}
onto its ``factors;'' namely, $\pi_X\colon X \times Y \to X$ given by
$\pi_X\colon (x,y)\mapsto x$ and and $\pi_Y\colon X \times Y \to Y$ given by $\pi_Y\colon
(x,y)\mapsto y$.


\eg\ $(\pi_X p, \pi_Y p) = p$ for any $p \in X \times Y$.

The Cartesian product is associative, in the sense that there is a
``natural'' isomorphism of sets between $X \times (Y \times Z)$ and $(X \times Y) \times
Z$. Because of this, we normally write simply $X \times Y \times Z$.


