\documentclass[12pt, a4paper]{article}
\usepackage[T1]{fontenc}
\usepackage{beton}
\usepackage{eulervm}
\DeclareFontSeriesDefault[rm]{bf}{sbc}
\usepackage[font=footnotesize]{caption}
\usepackage{ragged2e}
\usepackage{snotez}
\setsidenotes{
  text-format+=\RaggedRight}
\usepackage{amsmath}
\usepackage{microtype}
\usepackage[a4paper]{geometry}
\geometry{left=15mm,
  textwidth=0.6\paperwidth, % 126mm for A4
  textheight=0.6\paperheight}
%\usepackage{calc}
\geometry{marginparsep=5mm, marginparwidth=49mm}
\geometry{footskip=2in}
%%
\usepackage{tikz-cd}
%%
\setlength{\parskip}{\smallskipamount}
%%
\newcommand{\defn}[1]{\textbf{#1}}
\newcommand{\set}[1]{\mathbold{#1}}
\newcommand{\eg}{\emph{Example:}\relax}
\newcommand{\id}{\text{id}}
%\DeclareMathOperator{\id}{id}
%%
%%
\title{\vspace{-6ex}Sets}
\author{James Geddes}
\begin{document}
\maketitle
Set theory contains much of the vocabulary of
mathematics. There is not much content to the subject but there are
quite a few concepts and these concepts permeate the rest of
mathematics.

A \defn{set} is a “collection of things.” The term “collection” is not
defined. The things are called the \defn{elements} (or \defn{members})
of the set. To show that an element, $x$, is a member of some set,
$A$, write $x \in A$. A set is nothing more than the collection of its
elements: two sets are equal if and only if they have the same elements.

There are two important facts about sets: within a set, the elements
are unique; and there is no order to the elements.

To describe a concrete set with given elements, surround the elements with
braces. To describe a set containing those elements of another set, $X$, which
satisfy some predicate $P(x)$, write $\{x \in X \mid P(x)\}$. The set with no
elements is called the \defn{empty set} and written as $\emptyset$. There is only one
empty set.

\eg\ $\{0, 1, 2\}$ is the set containing 0, 1, and~2. It is the same
set as $\{2,0,1\}$.

\eg\ The expression $\{1, 1\}$ does not describe a set.

\eg\ $1\in\{0,1,2\}$. $3\notin\{0,1,2\}$.

\eg\ The set $\set{N}$ denotes the set of natural numbers,
$\{0,1,2,\dotsc\}$. The set $\{n \in \set{N} \mid \text{$n$ is prime}\}$ is
the set of primes.

\eg\ The set $\{\emptyset\}$ has one element (the empty set). The set $\{\emptyset,
\{\emptyset\}\}$ has two elements (the empty set, and the set containing the
empty set).

If every element of $B$ is also an element of $A$, we write $B \subset A$
and say that $B$ is a \defn{subset} of~$A$. The set of all subsets of
$A$ is called the \defn{power set} of $A$ and written $\mathcal{P}(A)$.

\eg\ The power set of $\{0,1,2\}$ has, as elements, $\emptyset$, $\{0\}$,
$\{1\}$, $\{2\}$, $\{0,1\}$, $\{0,2\}$, $\{1,2\}$, and $\{0,1,2\}\}$.

\eg\ $\emptyset \subset A$ for any $A$. Therefore, $\emptyset \in \mathcal{P}(A)$ for any $A$. 

The \defn{union} of two sets, $X$ and $Y$, written $X \cup Y$, is the set
of all elements of either $X$ or $Y$. The \defn{intersection} of two
sets, written $X \cap Y$, is the set of all elements in both $X$ and
$Y$. The \defn{set difference} of two sets, written $Y \setminus X$ (or
sometimes $Y-X$) is the set of all elements that are in $Y$ but not in
$X$.

Sometimes there is a set $\mathcal{U}$ that is understood to be the “set of all
things presently under consideration.” In that case, the
\defn{complement} of $X$, written $X^c$, is the set of all things in
$\mathcal{U}$ but not in $X$; that is, $X^c \equiv \mathcal{U} \setminus X$.

There are many identities involving these operations that follow from
the definitions.
\eg\ $X\cap(A\cup B) = (X\cap A) \cup (X\cap B)$.

\begin{sidefigure}
  \[\begin{tikzcd}
    X \arrow[r, "f"] & Y
  \end{tikzcd}\]
  \caption{One way to draw a map, $f\colon X\to Y$.}
\end{sidefigure}
Given two sets, $X$ and $Y$, a \defn{map}, $f\colon X \to Y$, is a rule
that assigns, to each element $x \in X$, an element $f(x) \in Y$. The set
$X$ is called the \defn{domain} of $f$ and the set $Y$ is called the
\defn{codomain} of~$f$. Maps are the principal means by which we compare
sets and much of the subject of set theory is really the study of
maps.\sidenote{The term “function” is sometimes used instead of “map;”
  sometimes “function” means a map where the codomain is a number;
  sometimes “map” means a certain kind of structure-preserving map.}

Sometimes we describe a map using the notation $f\colon x\mapsto f(x)$. (The
symbol $\mapsto$ is read as ``maps to.'')

\eg\ For $X$ any set, there is a unique map $\emptyset \to X$. (It may surprise
you that there is such a map at all. After all, there are no elements
in $\emptyset$ to be mapped to something in $X$. However, a map just has to
say what to do with each element of its domain and, since this domain
is empty, the “empty set of rules” is a perfectly fine map.)

\eg\ For any set $X$, the \emph{identity map}, $\id_X\colon X \to X$, is
the map given by $\id_X\colon x \mapsto x$ for every $x\in X$. Where the set
$X$ is implicitly known, we write simply “$\id$.”

A map is said to be \defn{one-to-one} (or \defn{injective}) if every
point in the domain is mapped to a \emph{unique} point in the
codomain.

The \defn{image} of $f$, written $f(X)$, is the set of all elements of
the codomain mapped to from \emph{some} element of the domain: $f(X) =
\{y \in Y \mid \text{$y = f(x)$ for some $x$}\}$. If the image of $f$ is
the entire codomain, the map is said to be \defn{onto} (or
\defn{surjective}).

\eg\ The identity map is both one-to-one and onto.

\eg\ The map $\set{R}\to\set{R}$ defined by $x\mapsto x^2$ is neither
one-to-one nor onto. It is not one-to-one because, for example, both
$2$ and $-2$ are mapped to $4$. It is not onto because the range is
only the non-negative reals.  

For two maps $f\colon B \to C$ and $g\colon A \to B$, their
\defn{composition}, $f\circ g\colon A \to C$ (read “$f$ after $g$”) is the
map defined by $(f\circ g)(x) = f(g(x))$.

\begin{sidefigure}
  \[
  \begin{tikzcd}
    A \arrow [rr, bend right, "f \circ g"' ] \arrow[r, "g"] & B \arrow[r, "f"] & C 
  \end{tikzcd}
  \]
  \caption{One way to visualise $f \circ g$. Notice that $f$ is applied
    after $g$.}
\end{sidefigure}

\eg\ $f\circ\id = \id\circ f = f$ for any map $f$. 

If, given $f\colon X \to Y$, there exists a $g\colon Y \to X$ such that
$f\circ g = \id_Y$ and $g\circ f = \id_X$, then we say that $f$ is an
\defn{isomorphism of sets} and that $X$ and $Y$ are isomorphic. The
map $g$ is called the \defn{inverse} of $f$ and written $g =
f^{-1}$. Isomorphic sets are “identical in every way that matters to
set theory.”

\eg\ A map $f$ is an isomorphism of sets if and only if it is both
one-to-one and onto. (If it is one-to-one, then we can “go backwards”
from any element in the range; if it is onto, then every element of
the codomain is in the range.)

\eg\ Two finite sets (sets with finitely many members) are isomorphic
if and only if the two sets have the same number of elements.

\eg\ (\defn{Cantor's theorem}) There is no isomorphism between $A$ and
$\mathcal{P}(A)$ for any set~$A$.

\eg\ (Georg Cantor) There is no isomorphism between the natural
numbers and the real numbers. In this, well-defined, sense, there are
``more'' real numbers than there are natural numbers: the real numbers
are ``uncountable.''

We sometimes do want to talk about collections of things where the
order matters. A \defn{pair} (sometimes “ordered pair”) is a
collection of two things where the order of the things matters and the
two things might be the same thing. For two things $x$ and $y$, a pair
is written $(x, y)$. We call $x$ the “first element” and $y$ the
“second element.” The main property of pairs is that $(x,y)=(p,q)$ if
and only if $x=p$ and $y=q$.

One might wonder whether a pair is a new kind of thing. In fact, pairs
can be defined in terms of sets. Here is one way to do so: $(x, y)\equiv
\{x, \{x,y\}\}$. (Though it's surprisingly subtle to say why this works.)

A \defn{finite tuple} is an ordered list of things, possibly
repeated. One may define a 3-tuple, $(a, b, c)$ say, as a pair whose
second element is a pair: $(a, (b, c))$.
  
Suppose $X$ and $Y$ are sets. Their \defn{Cartesian product}, $X \times Y$,
is the set of all pairs $(x, y)$ where $x \in X$ and $y \in Y$. The
Cartesian product comes with two natural \defn{projection maps} onto
its ``factors;'' namely, $\pi_X\colon X \times Y \to X$ given by $\pi_X\colon
(x,y)\mapsto x$ and and $\pi_Y\colon X \times Y \to Y$ given by $\pi_Y\colon (x,y)\mapsto y$.

\eg\ $\set{R}^2$ is the set of all pairs of real numbers. If you think
of $\set{R}$ as ``the real line,'' then you can think of $\set{R}^2$
as ``the plane.''

\eg\ $(\pi_X p, \pi_Y p) = p$ for any $p \in X \times Y$.

The Cartesian product is associative, in the sense that there is a
``natural'' isomorphism of sets between $X \times (Y \times Z)$ and $(X \times Y) \times
Z$. Because of this, we normally write simply $X \times Y \times Z$.



\end{document}
