\documentclass[10pt, a4paper, twocolumn]{article}
%\usepackage{graphicx}
%\usepackage{textcomp}
%\usepackage{amssymb}
%\usepackage{fontspec}
%\usepackage{minted}
\usepackage[T1]{fontenc}
\usepackage{concrete}
\usepackage{eulervm}
\usepackage{amsmath}
%\usepackage[amssymb,sansbold]{concmath}
\usepackage[margin=0.51in]{geometry}
%\setlength{\parskip}{\smallskipamount}
\usepackage{parskip}
\usepackage{booktabs}
%\usepackage{fancyhdr}
%%
\usepackage[style=authoryear]{biblatex}
\addbibresource{notes.bib}
%%
%%
\newcommand{\set}[1]{\mathbold{#1}}
\newcommand{\N}{\set{N}}
\newcommand{\pairs}{\set{P}}
\newcommand{\minus}[1]{\underline{#1}}
%%
\author{James Geddes}
\date{\today}
\title{Negative numbers}
\begin{document}%\maketitle
\maketitle

I assume you are familiar with the \emph{natural numbers}: 0, 1, 2, 3,
and so on.

One of the challenges in life is managing one's flock of sheep. The
natural numbers provide a way of marshalling and working with certain
facts about flocks. For example, one can count the number of sheep in
one's flock. Suppose, on Monday, I count my flock of sheep and
determine that I have 143 sheep. Then, on Tuesday I count again and
discover that there are 140: I may infer that I am missing some
sheep. What is more useful is that I am able to conclude that
precisely three sheep are missing.

The collection of these natural numbers is a set,
$\N=\{0,1,2,\dotsc\}$, which comes equipped with a map,
$+:\N\times\N\to\N$, known as \emph{addition}.\footnote{The notation
``$\N\times\N$'' means the set consisting of all pairs of natural
numbers. The notation ``$+:\N\times\N\to\N$'' means that $+$ is a rule
which assigns, to every pair of natural numbers, a natural number.}
Let us review some of the properties of the natural numbers.
\begin{enumerate}
\item There is a distinguished natural number, $0$, with the
  property that for $n$ any natural number, $n+0=n$. (We say that $0$
  is an ``identity'' for $+$.)\label{abmon:id}
\item For $l,m,n\in\N$ any three natural numbers, $(l+m)+n=l+(m+n)$. (We say
  that addition is ``associative''.)
\item For $m,n\in\N$ any two natural numbers, $m+n=n+m$. (We say that
  addition is ``commutative''.)
\end{enumerate}
Other properties of the natural numbers (but not all!) follow from
these three.\footnote{Mathematicians have a name for a structure
satisfying these three conditions: A set with an associative,
commutative, binary operation with an identity element is called an
\emph{abelian monoid}. However, we will not use this terminology nor
will we use any results about abelian monoids.} For example, it
follows that the element $0$ is unique, in the sense that if any other
element, say $0'$, has the property of (\ref{abmon:id}) above, then
$0'=0$. (\emph{Exercise:} Show this.)

The introduction gave an example of a real-world problem that can be
solved by use of the theory of natural numbers: Suppose I have 140
sheep. How many additional sheep must I locate in order to have 143
sheep? The question asks for a natural number, $n$, which makes the
following equation true:
\begin{equation*}
140 + n = 143.
\end{equation*}
With some practice, one can obtain the answer for small numbers ``by
inspection.'' In this case, the answer is $n=3$. For larger numbers,
schoolchildren are taught a method of obtaining the result by an
algorithm known as ``subtraction''. As a matter of notation, one
writes $n = 143 - 140$, but this is just notation: it means that $n$
satisfies the formula above.

Not all such problems have an answer! For example, the following
apparently similar problem does not have a solution:
\begin{equation*}
143 + n = 140. \quad\text{(No solution!)}
\end{equation*}
The answer would be written in the above notation as $n=140-143$ but there
is no such~$n$. Children are taught the rubric ``you can't
subtract a larger number from a smaller one,'' which has the same
content as the assertion that there is no natural number satisfying
the above equation.

Nevertheless, there are occasionally times when it would be convenient
to \emph{pretend} that we could find a natural number $n$ having the
property that $143+n=140$. Such a natural number isn't ``real,'' of
course, but pretending it exists can allow us to describe and solve
real problems in a way that is more notationally appealing than would
otherwise be the case. An example where this could prove useful is
that of \emph{debt}. If I own $140$ sheep but I owe you $143$, there's
a sense in which I will need another three sheep just to own no sheep!
Let us try to make these ideas more precise.

What might this $n$ look like if $143+n=140$? Observe that adding any natural
number to $143$ produces a natural number that is larger than
$143$. On the other hand, we want the result to be a natural number
that is \emph{smaller} than $143$---namely, $140$. But the smallest
natural number we could add to $143$ is $0$; so, in some sense, $n$
would have to be ``less than $0$.'' That is clearly a very strange
number of sheep!

We can say something more. We know that $143-140=3$. So in some sense
$n$ should be ``$3$ fewer than $0$.'' We will denote this strange
object by $\minus{3}$. Don't be fooled! Right now, there is no meaning
at all to $\minus{3}$: it's just a formal symbol. What we will do is
give a \emph{meaning} to the symbol $\minus{3}$ so that the expression
$143 + \minus{3} = 140$ ``makes sense.''

Our strategy will be as follows: we ``enlarge'' the set $\N$, to
include ``numbers less than zero, like $\minus{3}$ and $\minus{5}$;''
at the same time exending the operation of addition to these new
numbers so that the extended addition behaves in the way we expect
addition to behave. It is slightly tricky to do this in a way that
ensures we have included all the new numbers we need without
including ``unnecessary'' new numbers, all while making sure addition
does the right thing. In fact, we will first add ``too many'' new
numbers, in a way that makes it obvious that addition is behaving
appropriately, and then ``get rid of the redundancy.''

Let $\pairs$ be the set of all pairs of natural numbers: $\pairs =
\{(m, n) \mid m,n\in \N\}$. We introduce an operation
$\oplus:\pairs\times\pairs\to\pairs$ by the following rule:
\begin{equation}\label{pairs:addition}
  (a, b) \oplus (m, n) \equiv (a + m, b + n).
\end{equation}
That is: to ``add'' two elements of $\pairs$, add together their
corresponding components as natural numbers. This operation, $\oplus$, is
associative and commutative, and $(m,n)\oplus(0,0) = (m,n)$ for any
$(m,n)\in\pairs$. (Those properties follow immediately from the
corresponding properties of $+$.) Thus $\oplus$ ``behaves the way addition
is supposed to behave.''

Why did we define $\pairs$ this way? The interpretation is intended to
be something like this: If $m>n$, then $(m,n)$ should represent the
natural number $m-n$. For example, $(3,0)$ is supposed to represent
the natural number~$3$. On the other hand, if $m<n$, then $(m,n)$
represents the number that is ``less than zero by $n-m$.'' For
example, $(0,3)$ should represent the thing we wrote as $\minus{3}$
earlier. This interpretation nearly gives us what we were looking
for. For example---just following the rules---$(143,0) \oplus (0,3) =
(143,3)$. And that looks a lot like $143+\minus{3}=140$, which is what
we wanted to be able to write.

Unfortunately, this $\pairs$ is ``too large.'' The elements $(140,0)$
and $(143,3)$ are supposed to be ``the same number'' but they are
\emph{not} the same element of $\pairs$.

To fix this, we will say that $(m,n)$ and $(a,b)$ are
\emph{equivalent} if $m + b = a + n$.\footnote{We'd like to
write the condition for equivalence as $m-n=a-b$ but this expression
might not make sense; for example, when $m<n$.} We denote this equivalence by
writing $(m,n) \sim (a,b)$. For example $(143,3)\sim(140,0)$ because
$143+0=140+3$. 

Equivalent elements of $\pairs$ will have the interpretation as being
``the same number.'' The crucial property of equivalence is that it
plays nicely with the operation $\oplus$. For example, consider the
equation $5+3=8$. Translated into $\pairs$, this becomes
$(5,0)\oplus(3,0)=(8,0)$, which is true. But another way of writing $5$ is
$(6,1)$, because $(6,1)\sim(5,0)$, and another way of writing $3$ is
$(10,7)$, because $(10,7)\sim(3,0)$. Does the formula still work? Yes:
$(6,1)\oplus(10,7)=(16,8)$ and $(16,8)\sim(8,0)$.

The last step in our construction is therefore to make the previous
remarks more precise. We will ``identify equivalent elements of
$\pairs$, noting that $\oplus$ remains well-defined on the result.'' 

Let $\mathbold{n}$ be an element of $\pairs$. Consider the set of all
elements of $\pairs$ that are equivalent to $\mathbold{n}$. We call
this the \emph{equivalence class of $\mathbold{n}$} and denote it by
$\tilde{\mathbold{n}}$. That is: $\tilde{\mathbold{n}} \equiv \{
\mathbold{q}\in\pairs \mid \mathbold{q}\sim\mathbold{n}\}$. An element of an
equivalence class will be called a \emph{representative} of that
equivalence class.

Every element of $\pairs$ lies in one and only one equivalence
class. (To see this, note that if $\mathbold{n}\sim\mathbold{p}$ and
$\mathbold{n}\sim\mathbold{q}$ then $\mathbold{p}\sim\mathbold{q}$. In other
words, if $\mathbold{n}$ is equivalent to two other members of
$\pairs$, then those two members are in the same equivalence class.)
Furthermore, for any two $\mathbold{m},\mathbold{n}\in\pairs$, the
equivalence class of the sum $\mathbold{m}\oplus\mathbold{n}$ depends
\emph{only} on the equivalence classes of $\mathbold{m}$ and
$\mathbold{n}$. (This can be seen by working through the definition of
$\sim$.) Thus, we can write $\tilde{\mathbold{m}}\oplus\tilde{\mathbold{n}}$
to mean ``take any two representatives of $\tilde{\mathbold{m}}$ and
$\tilde{\mathbold{n}}$, add them together with $\oplus$, and take the
equivalence class of the result, noting that the result does not
depend on which representatives of $\tilde{\mathbold{m}}$ and
$\tilde{\mathbold{n}}$ were chosen.''

We denote by $\pairs/{\sim}$ the set of equivalence classes; and we note
that $\oplus$ is well-defined on these equivalence classes.

Now we can finally construct the set we seek. By the \emph{integers},
$(\set{Z}, +)$, we mean the set $P/{\sim}$ together with the operation
$\oplus$ on equivalence classes, which we write as~$+$.

\emph{Remarks}. At this point---following the big definition---you would
normally be advised to try a few calculations in $\set{Z}$ to get the
hang of this new mathematical object. These calculations will be a
little tedious: you have to pick representatives of equivalence
classes, do the calculations, and then decide whether the result is
the representative you acually want.

However, there is a simplification available which will make your life
significantly easier: It turns out that any equivalence class of
$\pairs$ will \emph{either} have a representative of the form $(m, 0)$
for some $m$ \emph{or} a representative of the form $(0, n)$ for some
$n$. (\emph{Exercise:} Convince yourself that this is true.) That is,
you only have to worry about numbers like $(3,0)$ and
$(0,5)$---everything else is equivalent to one of these.

The numbers of the form $(m,0)$ behave exactly like the ordinary
natural numbers $m$, and it is common to write the equivalence class
of $(5,0)$ just as $5$. We call these \emph{positive numbers}. The
numbers of the form $(0,n)$ also behave like the natural numbers, and
we write the equivalence class of, say, $(0,3)$ as $\minus{3}$. We
call numbers of the form $(0,n)$ \emph{negative numbers}. The content
of the simplification is that you only need to worry about postive
numbers like $5$ and negative numbers like $\minus{3}$.

\emph{Example}. We can now answer the seemingly-impossible problem we
posed at the beginning: find a ``number'' $n$ such that
$143+n=140$. The answer is $n=\minus{3}$. \emph{Proof:} $\minus{3}$
means the equivalence class in $\pairs$ having representative $(0,3)$,
and $143$ means the equivalence class having representative
$(143,0)$. Whence,
\begin{equation*}
(143,0) \oplus (0,3) = (143,3),
\end{equation*}
and, since $(143,3)\sim(140,0)$, we write the answer as simply $140$.

\emph{Exercise:} What is $6-10$? (Note: the notation $x = m - n$ means
$x+n=m$, so this exercise asks for a number which, when added to $10$,
gives $6$.)

\emph{Exercise:} Show that $m+\minus{m}=0$ for any positive number $m$.

\emph{Exercise:} Define the map $-:\set{Z}\to\set{Z}$ by $-m\equiv 0-m$. Show
that, for $m$ any positive number, $\minus{m}=-m$. 

\emph{Exercise:} Show that, in general, $m-n=m+(-n)$. (Because of these
results, it is usual to write $-5$ rather than $\minus{5}$.)













\end{document}
