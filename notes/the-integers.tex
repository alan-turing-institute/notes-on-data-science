\documentclass[10pt, a4paper, twocolumn]{article}
%\usepackage{graphicx}
%\usepackage{textcomp}
%\usepackage{amssymb}
%\usepackage{fontspec}
%\usepackage{minted}
\usepackage[T1]{fontenc}
\usepackage{concrete}
\usepackage{eulervm}
\usepackage{amsmath}
%\usepackage[amssymb,sansbold]{concmath}
\usepackage[margin=0.51in]{geometry}
%\setlength{\parskip}{\smallskipamount}
\usepackage{parskip}
\usepackage{booktabs}
%\usepackage{fancyhdr}
%%
\usepackage[style=authoryear]{biblatex}
\addbibresource{notes.bib}
%%
%%
\newcommand{\N}{\mathbold{N}}
\author{James Geddes}
\date{\today}
\title{The Integers}
\begin{document}%\maketitle
\section*{The Integers}

I assume you are familiar with the \emph{natural numbers}: 0, 1, 2, 3,
and so on.

One of the challenges of life is managing ones flock of sheep. The
natural numbers provides a way of marshalling and working with certain
facts about flocks. For example, if one determines that on Tuesday one
has, say, 143 sheep but, come Wednesday, there are 142, then one may
infer that one sheep is missing. If, on the other hand, there are 144
sheep on Wednesday, then perhaps one of Owain's has got through the
hole in the fence again.

The collection of these natural numbers is a set,
$\N=\{0,1,2,\dotsc\}$, and they come equipped with a map,
$+:\N\times\N\to\N$, known as \emph{addition}.\footnote{The notation
``$\N\times\N$'' means the set consisting of all pairs of natural
numbers. The notation ``$+:\N\times\N\to\N$'' means that $+$ is a rule
which assigns, to every pair of natural numbers, a natural number.}
Let us review some of the properties of the natural numbers.
\begin{enumerate}
\item There is a distinguished natural number, $0$, with the
  property that for $n$ any natural number, $n+0=n$.
\item For $l,m,n\in\N$ any three natural numbers, $(l+m)+n=l+(m+n)$. (We say
  that addition is ``associative''.)
\item For $m,n\in\N$ any two natural numbers, $m+n=n+m$. (We say that
  addition is ``commutative''.)
\end{enumerate}
Other properties of the natural numbers (but not all!) follow from
these three.\footnote{Mathematicians have a name for a structure
satisfying these three conditions: A set with an associative,
commutative, binary operation having an identity element is called an
\emph{abelian monoid}. However, we will not use this terminology nor
will we use any results about abelian monoids.} For example,
it can be shown that the element $0$ mentioned in (1) is unique, in
the sense that if any other element, say $0'$, has the same property,
then $0'=0$. (Exercise: show this.)

Here is an example of a real-world problem that can be solved with the
use of the theory of natural numbers. Suppose I have five sheep. How
many additional sheep must I acquire in order to have eight sheep? The
question asks for a natural number, $n$, which makes the following
equation true:
\begin{equation*}
5 + n = 8.
\end{equation*}
With some practice, one can obtain the answer for small numbers ``by
inspection.'' In this case, it is $n=3$. For larger numbers,
schoolchildren are taught a method of obtaining the result by an
algorithm known as ``subtraction''. As a matter of notation, one
writes $n = 8 - 5$, which just means that $n$ satisfies the formula
above but shows when to apply the ``subtraction'' algorithm.

Not all such problems have an answer. For example, the following
problem does not have a solution:
\begin{equation*}
8 + n = 5. \quad\text{(No solution!)}
\end{equation*}
The answer would be written in the above notation as $5-8$ but
children are taught the rubric that ``you can't subtract a larger
number from a smaller one,'' which has the same content as the
assertion that there is no natural number satisfying the above
equation.

Nonetheless, there are occasionally times when it would be convenient
to pretend that we \emph{could} find a natural number $n$ having the
property that $8+n=5$. Such a natural number isn't ``real,'' of
course, but pretending it exists can allow us to describe and solve
real problems in a way that is more notationally appealing than would
otherwise be the case.

Before describing an example of this kind of problem, let us try to
say something about what this $n$ might ``look like.'' Note that
adding any natural number to $8$ produces a natural number that is
larger than $8$. We want the result to be a natural number, $5$, that
is \emph{smaller} than $8$. But the smallest natural number we could
add is $0$; so, in some sense, $n$ would have to be ``less than 0.''
That is clearly a very strange number of sheep!

The problem that will motivate our attempt to introduce ``numbers''
that are ``less than zero'' may in fact be familiar to you: it is the
problem of managing \emph{debt}. Suppose that my flock consists of
eight sheep but three of them are \emph{owed} to you, perhaps through
an imprudent bet. There is a sense in which I physically ``have''
eight sheep \emph{and} I ``have'' a debt of three sheep, so I
\emph{own} only five sheep. The debt of three sheep, when ``added'' to
the eight sheep in my flock, seems to \emph{reduce} the number of
sheep that count as my total wealth.









\end{document}
